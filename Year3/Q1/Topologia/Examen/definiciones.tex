\documentclass[12pt]{article}
\usepackage[a4paper,margin=2.5cm]{geometry}
\usepackage{amsmath, amsfonts, amssymb, amsthm}
\usepackage{physics}
\usepackage{mathtools}
\usepackage[hidelinks]{hyperref}

\usepackage[most]{tcolorbox}
\tcbset{
  definitionstyle/.style={
    colback=gray!10,    % color de fondo
    colframe=gray!50,   % color del borde
    boxrule=0.4pt,      % grosor del borde
    arc=3pt,            % redondeo de esquinas
    left=6pt,right=6pt,top=4pt,bottom=4pt,
  }
}


\title{Definiciones}
\author{Juan Rodríguez}
\date{}
\begin{document}
\maketitle
\tableofcontents
\newpage
\phantomsection
\section*{Métrica o Distancia}
\addcontentsline{toc}{section}{Métrica o Distancia}
\begin{tcolorbox}[definitionstyle]
Sea \( X \) un conjunto.  
Se dice que \( d : X \times X \to \mathbb{R} \) define una \textbf{distancia} (o \textbf{métrica}) en \( X \) si se cumplen las siguientes propiedades:

\begin{enumerate}
    \item \( d(x,y) = 0 \) si y solo si \( x = y \) \quad (Reflexiva)
    \item \( d(x,y) = d(y,x) \) \quad (Simetría)
    \item \( d(x,y) \leq d(x,z) + d(z,y) \) para todo \( x,y,z \in X \) \quad (Desigualdad triangular)
\end{enumerate}

En estas condiciones, el par \( (X,d) \) se denomina \textbf{espacio métrico}.
\end{tcolorbox}

\section*{Continuidad}
\addcontentsline{toc}{section}{Continuidad}
\subsection*{Función continua en espacios métricos}
\addcontentsline{toc}{subsection}{Función continua en espacios métricos}
\begin{tcolorbox}[definitionstyle]
Sean \((X_1, d_1)\) y \((X_2, d_2)\) espacios métricos.  
La función \(f : X_1 \to X_2\) se dice \textbf{continua en un punto} \(x \in X_1\) si y solo si:
\[
\forall \varepsilon > 0, \ \exists \delta > 0 \text{ tal que } 
f(B_\delta(x)) \subseteq B_\varepsilon(f(x)).
\]
La función \(f\) se llama \textbf{continua} si lo es en todos los puntos de su dominio.
\end{tcolorbox}

\subsection*{Caracterización topológica de continuidad}
\addcontentsline{toc}{subsection}{Caracterización topológica de continuidad}
\begin{tcolorbox}[definitionstyle]
Sea \(f : X_1 \to X_2\) una función entre espacios topológicos.  
La función \(f\) es \textbf{continua} si y solo si la preimagen de todo conjunto abierto de \(X_2\) es abierta en \(X_1\); es decir,
\[
\forall A \subseteq X_2 \text{ abierto}, \quad f^{-1}(A) \text{ es abierto en } X_1.
\]
\end{tcolorbox}
\section*{Espacios topológicos}
\addcontentsline{toc}{section}{Espacios topológicos}
\subsection*{Topología}
\addcontentsline{toc}{subsection}{Topología}
\begin{tcolorbox}[definitionstyle]
Dado un conjunto \( X \), se dice que una colección de subconjuntos \( \mathcal{T} \) de \( X \) es una \textbf{topología} si se cumplen las siguientes propiedades:

\begin{enumerate}
    \item \( \varnothing, X \in \mathcal{T} \)
    \item La intersección de un número \textbf{finito} de elementos de \( \mathcal{T} \) también pertenece a \( \mathcal{T} \):
    \[
        \bigcap_{i=1}^{k} T_i \in \mathcal{T}
    \]
    \item La unión de cualquier número (posiblemente infinito) de elementos de \( \mathcal{T} \) también pertenece a \( \mathcal{T} \):
    \[
        \bigcup_{i \in I} T_i \in \mathcal{T}
    \]
\end{enumerate}

La pareja \( (X, \mathcal{T}) \) se llama \textbf{espacio topológico}, y los elementos de \( \mathcal{T} \) se denominan \textbf{abiertos}.
\end{tcolorbox}

\subsection*{Base de una topología}
\addcontentsline{toc}{subsection}{Base de una topología}
\begin{tcolorbox}[definitionstyle]
Una colección de subconjuntos \( B_i \) de \( X \) se llama \textbf{base} si:
\[
\forall x \in X,\ \exists B_i \in \mathcal{B} : x \in B_i
\]
Además, si un punto está en la intersección de dos elementos, hay un elemento de la base en la
intersección que contiene este punto 
\[
\forall x \in B_1 \cap B_2,\ \exists B_3 \in \mathcal{B} : x \in B_3 \subseteq B_1 \cap B_2.
\]
\end{tcolorbox}
\subsection*{Topología heredada}
\addcontentsline{toc}{subsection}{Topología heredada}
\begin{tcolorbox}[definitionstyle]
Si tenemos un espacio topológico \( (X, \mathcal{T}) \) y un subconjunto \( A \subseteq X \), podemos definir una topología en \( A \) dada por:
\[
\mathcal{T}_A = \{\, A \cap T : T \in \mathcal{T} \,\}.
\]
La pareja \( (A, \mathcal{T}_A) \) se llama \textbf{topología heredada} de \( X \).
\end{tcolorbox}
\subsection*{Topología inducida por una distancia}
\addcontentsline{toc}{subsection}{Topología inducida por una distancia}
\begin{tcolorbox}[definitionstyle]  
Cualquier función de distancia \( d \) en un conjunto \( X \) permite definir \textbf{bolas abiertas} como los conjuntos de puntos que están a una distancia menor que una dada respecto de un centro:
\[
B_r(x) = \{\, y \in X : d(x,y) < r \,\}.
\]
Si tomamos estas bolas abiertas como base de nuestra topología, la topología resultante se llama \textbf{topología inducida por una distancia}.
\end{tcolorbox}
\subsection*{Topología del orden}
\addcontentsline{toc}{subsection}{Topología del orden}
\begin{tcolorbox}[definitionstyle] 
Dado un conjunto ordenado \( X \), la \textbf{topología del orden} es la generada por la base de intervalos abiertos de la forma \( (a, b) = \{\, x \in X : a < x < b \,\} \),  
añadiendo además los conjuntos \( [\min(X), b) \) y \( (a, \max(X)] \) si estos extremos existen en \( X \).
\end{tcolorbox}
\subsection*{Topología producto}
\addcontentsline{toc}{subsection}{Topología producto}
\begin{tcolorbox}[definitionstyle] 
Si \( (X, \mathcal{T}_1) \) y \( (Y, \mathcal{T}_2) \) son espacios topológicos, se llama \textbf{topología producto} en \( X \times Y \) a la topología cuya base está formada por todos los productos cartesianos de conjuntos abiertos de los espacios originales:
\[
\mathcal{B} = \{\, U \times V : U \in \mathcal{T}_1,\ V \in \mathcal{T}_2 \,\}.
\]
\end{tcolorbox}
\section*{Conceptos básicos}
\addcontentsline{toc}{section}{Conceptos básicos}
\subsection*{Conjunto abierto}
\addcontentsline{toc}{subsection}{Conjunto abierto}
\begin{tcolorbox}[definitionstyle]
Un conjunto \( A \subseteq X \) se llama \textbf{abierto} si para cada punto \( x \in A \) existe un entorno abierto \( U \) tal que \( x \in U \subseteq A \).
\end{tcolorbox}
\subsection*{Conjunto cerrado}
\addcontentsline{toc}{subsection}{Conjunto cerrado}
\begin{tcolorbox}[definitionstyle]
Un conjunto \( A \subseteq X \) se llama \textbf{cerrado} si contiene a todos sus puntos frontera, o equivalentemente, si su complementario \( X \setminus A \) es abierto.
\end{tcolorbox}
\subsection*{Punto fronterizo y frontera}
\addcontentsline{toc}{subsection}{Punto fronterizo y frontera}
\begin{tcolorbox}[definitionstyle]
Un punto \( x \in X \) se llama \textbf{fronterizo} de un conjunto \( A \) si todo entorno de \( x \) contiene puntos de \( A \) y de su complementario \( X \setminus A \).  
El conjunto de todos los puntos fronterizos de \( A \) se llama la \textbf{frontera} de \( A \), denotada \( \operatorname{Fr}(A) \).
\end{tcolorbox}
\subsection*{Interior}
\addcontentsline{toc}{subsection}{Interior}
\begin{tcolorbox}[definitionstyle]
 El \textbf{interior} de un conjunto \( A \), denotado \( \operatorname{Int}(A) \), es el mayor conjunto abierto contenido en \( A \).  
Equivalentemente, es la unión de todos los abiertos contenidos en \( A \).
\end{tcolorbox}
\subsection*{Clausura}
\addcontentsline{toc}{subsection}{Clausura}
\begin{tcolorbox}[definitionstyle]
La \textbf{clausura} de un conjunto \( A \), denotada \( \overline{A} \), es el menor conjunto cerrado que contiene a \( A \).  
Se puede expresar como \( \overline{A} = A \cup \operatorname{Fr}(A) \).
\end{tcolorbox}
\subsection*{Puntos de acumulación}
\addcontentsline{toc}{subsection}{Puntos de acumulación}
\begin{tcolorbox}[definitionstyle]
Un punto \( x \in X \) se llama \textbf{punto de acumulación} (o límite) de un conjunto \( A \) si todo entorno de \( x \) contiene algún punto de \( A \) distinto de \( x \).  
El conjunto de todos los puntos de acumulación de \( A \) se denota por \( A' \).
\end{tcolorbox}
\subsection*{Puntos aislados}
\addcontentsline{toc}{subsection}{Puntos aislados}
\begin{tcolorbox}[definitionstyle]
Los \textbf{puntos aislados} de un conjunto \( A \) son aquellos que pertenecen a \( \overline{A} \) pero no son puntos de acumulación, es decir, los elementos de \( \overline{A} \setminus A' \).
\end{tcolorbox}
\section*{Densidad}
\addcontentsline{toc}{section}{Densidad}
\subsection*{Conjunto denso}
\addcontentsline{toc}{subsection}{Conjunto denso}
\begin{tcolorbox}[definitionstyle]
Un subconjunto \( H \subseteq X \) se llama \textbf{denso} en \( X \) si su clausura es todo el espacio, es decir:
\[
\overline{H} = X.
\]
Equivalentemente, \( H \) es denso si su intersección con cualquier abierto no vacío de \( X \) es no vacía.  
Intuitivamente, los puntos de \( H \) se aproximan arbitrariamente a cualquier punto de \( X \).
\end{tcolorbox}
\subsection*{Conjunto no denso en ninguna parte}
\addcontentsline{toc}{subsection}{Conjunto no denso en ninguna parte}
\begin{tcolorbox}[definitionstyle]
Un subconjunto \( A \subseteq X \) se llama \textbf{no denso en ninguna parte} si el interior de su clausura es vacío:
\[
\operatorname{Int}(\overline{A}) = \varnothing.
\]
Esto significa que \( A \) sólo puede ser frontera.
\end{tcolorbox}
\section*{Axiomas de Separación $T$}
\addcontentsline{toc}{section}{Axiomas de Separación T}
\begin{tcolorbox}[definitionstyle]
\subsection*{Propiedad de Fréchet (T\textsubscript{1})}
\addcontentsline{toc}{subsection}{Propiedad de Fréchet (T1)}  
Un espacio topológico \( X \) es \textbf{T\textsubscript{1}} si para cada par de puntos distintos \( x_1, x_2 \in X \),  
existen entornos abiertos \( U_1, U_2 \subseteq X \) tales que:
\[
x_1 \in U_1, \quad x_2 \notin U_1, \qquad \text{y} \qquad x_2 \in U_2, \quad x_1 \notin U_2.
\]
Equivalente a decir que todos los puntos de \( X \) son conjuntos cerrados.
\end{tcolorbox}
\subsection*{Propiedad de Hausdorff (T\textsubscript{2})}
\addcontentsline{toc}{subsection}{Propiedad de Hausdorff (T2)}
\begin{tcolorbox}[definitionstyle]
Un espacio topológico \( X \) es \textbf{Hausdorff} o \textbf{T\textsubscript{2}} si para cada par de puntos distintos \( x_1, x_2 \in X \),  
existen entornos abiertos disjuntos \( U_1, U_2 \subseteq X \) tales que:
\[
x_1 \in U_1, \quad x_2 \in U_2, \quad \text{y} \quad U_1 \cap U_2 = \varnothing.
\]
Intuitivamente, los puntos de un espacio Hausdorff “viven en casas separadas”.
\end{tcolorbox}
\section*{Axiomas de Numerabilidad}
\addcontentsline{toc}{section}{Axiomas de Numerabilidad}
\subsection*{Primer Axioma de Numerabilidad (1AN)}
\addcontentsline{toc}{subsection}{Primer Axioma de Numerabilidad (1AN)}
\begin{tcolorbox}[definitionstyle]
Un espacio topológico \( X \) satisface el \textbf{primer axioma de numerabilidad} si para cada punto \( x \in X \)  
existe una colección numerable de entornos \( \{A_i(x)\}_{i \in \mathbb{N}} \) tal que  
todo entorno abierto \( B(x) \) de \( x \) contiene al menos uno de los \( A_i(x) \).
 
Todo espacio metrizable cumple el primer axioma de numerabilidad.
\end{tcolorbox}
\subsection*{Segundo Axioma de Numerabilidad (2AN)}
\addcontentsline{toc}{subsection}{Segundo Axioma de Numerabilidad (2AN)}
\begin{tcolorbox}[definitionstyle]
Un espacio topológico \( X \) satisface el \textbf{segundo axioma de numerabilidad}  
si la topología de \( X \) tiene una base numerable, es decir,  
existe una familia numerable de abiertos \( \{B_i\}_{i \in \mathbb{N}} \) tal que  
todo abierto de \( X \) puede expresarse como unión de algunos de los \( B_i \).
\end{tcolorbox}
\section*{Homeomorfismo}
\addcontentsline{toc}{section}{Homeomorfismo}
\begin{tcolorbox}[definitionstyle]
Sea \( f : X \to Y \) una biyección entre espacios topológicos.  
Se dice que \( f \) es un \textbf{homeomorfismo} si \( f \) es continua y abierta  
(es decir, si la imagen de cualquier conjunto abierto de \( X \) es abierta en \( Y \)).

Equivalentemente, \( f \) es homeomorfismo si \( f \) y su inversa \( f^{-1} \) son continuas.

En tal caso, los espacios \( X \) y \( Y \) se dicen \textbf{homeomorfos},  
lo que significa que son topológicamente equivalentes.
\end{tcolorbox}
\end{document}