\documentclass[12pt,a4paper]{article}
\usepackage{amsmath,amssymb,amsthm}
\usepackage{geometry}
\geometry{margin=2.5cm}

\title{Hojas de Ejercicios de Topología}
\author{}
\date{}

\begin{document}

\maketitle
\setlength{\parskip}{0.8em}

%%%%%%%%%%%%%%%%%%%%%%%%%%%%%%%%%%%%%%%%%%%%%%%%%%%%%%%%%%%%%%
% Topología – Hojas 7 y 8
%%%%%%%%%%%%%%%%%%%%%%%%%%%%%%%%%%%%%%%%%%%%%%%%%%%%%%%%%%%%%%

\section*{Tema 7. Espacios conexos}

\subsection*{Ejercicio 1}
  Considera el conjunto
  \[
      A = \Big([0,1] \times \bigcup_{n \in \mathbb{N}} \Big\{\frac{1}{n}\Big\}\Big) 
      \,\cup\, \{0\} \times (0,1) \subset \mathbb{R}^2.
  \]
  \begin{enumerate}
      \item Dibújalo.
      \item Busca su clausura.
      \item ¿Es conexo?
      \item ¿Es conexo por caminos?
  \end{enumerate}

  \vspace{10cm}

  \subsection*{Ejercicio 2}
  Construye un espacio en el plano que sea conexo por caminos pero no conexo localmente.

  \vspace{10cm}

  \subsection*{Ejercicio 3}
  Sea
  \[
     E = \left\{(x,y)\in\mathbb{R}^2 : x = 1 - \frac{1}{t},\ t\ge 1 \right\}
     \,\cup\, \{(0,\sin t) : t\in\mathbb{R}\}.
  \]
  ¿Es conexo? ¿Es conexo por caminos?

  \vspace{10cm}

  \subsection*{Ejercicio 4}
  ¿Cómo son los subconjuntos conexos de $\mathbb{R}$ con la topología cofinita?
  \begin{enumerate}
      \item Nombra 3 subconjuntos no conexos y 2 conexos.
      \item Describe todos los subconjuntos conexos.
  \end{enumerate}

  \vspace{10cm}

  \subsection*{Ejercicio 5}
  Sea $N \subset \mathbb{R}^2$ un conjunto numerable. Demuestra que $\mathbb{R}^2 \setminus N$ es conexo por caminos.

  \vspace{10cm}

  \subsection*{Ejercicio 6}
  Demuestra que todo subconjunto conexo de $\mathbb{R}^2$ que tiene más de un punto es no numerable.

  \vspace{10cm}

  \subsection*{Ejercicio 7}
  Demuestra que el conjunto de Cantor es totalmente disconexo.

  \vspace{10cm}

\section*{Tema 8. Espacios compactos}

  \subsection*{Ejercicio 1}
  Indica si es compacto
  \[
      E = \{\,(-1)^n + \tfrac{1}{n} : n \in \mathbb{N}\,\} \subset \mathbb{R}
  \]
  con la topología canónica.

  \vspace{10cm}

  \subsection*{Ejercicio 2}
  Indica si es compacto
  \[
     A = [0,1] \times \{0.5\}
  \]
  con la topología del orden lexicográfico en $\mathbb{R}^2$.

  \vspace{10cm}

  \subsection*{Ejercicio 3}
  Sea $X$ un conjunto con la topología discreta. ¿Cuándo es compacto?

  \vspace{10cm}

  \subsection*{Ejercicio 4}
  Inventa un espacio topológico $(X,\mathcal{T})$ que tenga un subconjunto compacto no cerrado.

  \vspace{10cm}

  \subsection*{Ejercicio 5}
  ¿Es cierto que la unión de una familia de compactos siempre es compacta?

  \vspace{10cm}

  \subsection*{Ejercicio 6}
  Considera la base
  \[
     \mathcal{B} = \big\{\,\{0,n\} : n\in\mathbb{Z}\,\big\}
  \]
  que genera una topología en $\mathbb{Z}$.
  \begin{enumerate}
      \item ¿Es compacto $A = \{0\}$?
      \item ¿Cuál es $\overline{A}$?
      \item ¿Es compacto $\overline{A}$?
  \end{enumerate}

  \vspace{10cm}

  \subsection*{Ejercicio 7}
  Demuestra que una biyección $X \to Y$ tal que $X$ es compacto e $Y$ es Hausdorff es un homeomorfismo.  

  \textit{Pista: demuestra que pasa cerrados a cerrados.}

  \vspace{10cm}

  \subsection*{Ejercicio 8}
  ¿Son compactos con la topología del orden:
  \begin{enumerate}
      \item $[0,1] \times [0,1]$?
      \item $[0,1] \times (0,1]$?
  \end{enumerate}

\end{document}