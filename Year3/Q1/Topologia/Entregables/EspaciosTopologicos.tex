\documentclass[12pt]{article}
\usepackage[a4paper,margin=2.5cm]{geometry}
\usepackage{amsmath, amsfonts, amssymb, amsthm}
\usepackage{physics}
\usepackage{mathtools}
\usepackage{tikz}
\usetikzlibrary{angles,quotes}
\title{Entregables Espacios Topologicos Topología}
\author{Juan Rodríguez}
\date{}
\begin{document}
\maketitle
\section*{Ejercicio 1: ¿Son una topología?}

\begin{enumerate}
\item $X=\{a,b,c\}$, \ 
$T_1=\{\varnothing,\{a\},\{b\},\{a,c\},\{a,b,c\},\{a,b\}\}$.

\textbf{Solución.} 
Comprobamos axiomas:
(i) $\varnothing,X\in T_1$ (sí).
(ii) Intersección finita: por ejemplo,
$\{a\}\cap\{b\}=\varnothing\in T_1$,
$\{a\}\cap\{a,c\}=\{a\}\in T_1$,
$\{a,b\}\cap\{a,c\}=\{a\}\in T_1$, etc.
(iii) Unión arbitraria: 
$\{a\}\cup\{b\}=\{a,b\}\in T_1$,
$\{b\}\cup\{a,c\}=X\in T_1$, etc.
Como todos los casos cierran, 
\[
\boxed{T_1\text{ es topología en }X.}
\]

\item $X=\mathbb{R}$, \ 
$\mathcal{T}=\{(a,+\infty):a\in\mathbb{R}\}$.

\textbf{Solución.}
No es topología tal como está escrita porque 
\(\varnothing\notin\mathcal{T}\).
(Obs.: estas semirrectas forman \emph{base} de una topología —la de semirrectas derechas—,
pero la familia dada no incluye \(\varnothing\) ni garantiza que toda unión de ellas
siga estando en \(\mathcal{T}\) como \emph{un solo elemento}.) 
\[
\boxed{\mathcal{T}\text{ no es topología (falta } \varnothing).}
\]

\item $X=\mathbb{R}$, \ 
$\mathcal{S}=\{[a,b):a,b\in\mathbb{R},\,a<b\}$.

\textbf{Solución.}
Tampoco es topología como \emph{colección} final, porque (i) 
\(\varnothing\notin\mathcal{S}\) ni \(X\in\mathcal{S}\); 
(ii) una unión arbitraria de intervalos de la forma $[a,b)$ 
no tiene por qué ser nuevamente un único intervalo $[a,b)$.
(En realidad, $\mathcal{S}$ es \emph{base} de la topología de Sorgenfrey.) 
\[
\boxed{\mathcal{S}\text{ no es topología (sí es base).}}
\]
\end{enumerate}

\section*{Ejercicio 2: ¿Son una topología?}

\begin{enumerate}
\item $X=\{a,b,c\}$, \ 
$T_2=\{\varnothing,\{a\},\{c\},\{b,c\},\{a,b,c\},\{a,b\}\}$.

\textbf{Solución.}
No cierra por uniones: 
\(\{a\}\cup\{c\}=\{a,c\}\notin T_2\).
\[
\boxed{T_2\text{ no es topología.}}
\]

\item $X=\mathbb{R}$, \ 
$\mathcal{C}=\{U\subset\mathbb{R}:\ |\mathbb{R}\setminus U|<\infty\}$.

\textbf{Solución.}
La \emph{topología cofinita} estándar es 
\(\mathcal{C}\cup\{\varnothing\}\).
Tal como está escrita (sin $\varnothing$ explícita),
\(\varnothing\notin\mathcal{C}\) porque 
\(|\mathbb{R}\setminus\varnothing|=|\mathbb{R}|=\infty\).
Luego la familia dada \emph{no} es topología,
aunque basta añadir $\varnothing$ para que sí lo sea (y, de hecho, lo es):
las uniones arbitrarias de cofinito siguen siendo cofinito,
y las intersecciones finitas de cofinito también lo son.
\[
\boxed{\mathcal{C}\ \text{no es topología tal como está (falta }\varnothing).}
\]
\end{enumerate}

\section*{Ejercicio 3}
\emph{Halla un ejemplo de topología heredada de un espacio que no coincida con la topología “usual” (interior/inferior).}

\textbf{Ejemplo claro con Sorgenfrey.}
Sea \( (\mathbb{R},\tau_S) \) la topología de Sorgenfrey (base \( [a,b) \)).
Considérese el \emph{subespacio} \(X=\mathbb{R}\subset(\mathbb{R},\tau_S)\):
la topología \emph{heredada} en \(X\) \emph{coincide} con \(\tau_S\),
pero \emph{no} coincide con la topología usual canonica \(\tau_C\) en \(\mathbb{R}\).
Por ejemplo, \( [0,1) \) es abierto en \(\tau_S\) y no es abierto en \(\tau_C\).
Así, en el mismo conjunto \(X=\mathbb{R}\) tenemos dos topologías distintas:
la heredada de Sorgenfrey y la canonica.

\[
\boxed{\text{Conclusión: la topología heredada de Sorgenfrey en }X
\text{ no coincide con la canonica de }X.}
\]

\subsection*{Ejercicio 4}
Determinar el interior, la frontera y la clausura de los siguientes conjuntos.

\begin{enumerate}
\item $(0,2)$ en $\mathbb{R}$ (topología usual).\\[4pt]
$\operatorname{Int}((0,2)) = (0,2)$, \quad 
$\overline{(0,2)} = [0,2]$, \quad 
$Fr((0,2)) = \{0,2\}.$

\item $\{1/n : n \in \mathbb{Z}^+\}$ en $\mathbb{R}$.\\[4pt]
Todo punto $1/n$ es aislado $\rightarrow \operatorname{Int}(A) = \varnothing$. 
El punto $0$ es límite porque $1/n \to 0$.\\
$\overline{A} = \{1/n : n \in \mathbb{Z}^+\} \cup \{0\}$, \quad
$Fr(A) = \{1/n : n \in \mathbb{Z}^+\} \cup \{0\}.$

\item $(0,2)$ en $(0,4)$ con la topología subespacio.\\[4pt]
$\operatorname{Int}(A) = (0,2)$, \quad
$\overline{A} = (0,2]$, \quad
$Fr(A) = \{2\}$.
\end{enumerate}

\subsection*{Ejercicio 5}
Determinar el interior, la frontera y la clausura de los siguientes conjuntos.

\begin{enumerate}
\item $A = \{-3-\tfrac{1}{n} : n\in\mathbb{N}\} \cup (1,2) \cup \{4+\tfrac{1}{n} : n\in\mathbb{N}\}$ \\
en la recta de Sorgenfrey (base de abiertos $[a,b)$).

\begin{itemize}
\item \textbf{Interior:} las partes discretas no contienen abiertos, mientras que $(1,2)$ sí es abierto.\\
$\operatorname{Int}_{\mathbb{S}}(A) = (1,2).$
\item \textbf{Clausura:} $\overline{(1,2)}^{\,\mathbb{S}} = [1,2)$, \quad 
el punto $4$ se añade porque $[4,4+\varepsilon)$ corta $\{4+\tfrac1n\}$, 
mientras que $-3$ no pertenece a la clausura (no hay puntos de $A$ a su derecha).\\[2pt]
$\displaystyle \overline{A}^{\,\mathbb{S}} = 
\{-3-\tfrac{1}{n} : n\in\mathbb{N}\} \cup [1,2) \cup 
\{4+\tfrac{1}{n} : n\in\mathbb{N}\} \cup \{4\}.$
\item \textbf{Frontera:}\\
$Fr(A) = \{-3-\tfrac{1}{n}\} \cup \{1\} \cup \{4+\tfrac{1}{n}\} \cup \{4\}.$
\end{itemize}

\item $[1,2]\cup\{3\}$ en $\mathbb{R}$ (topología usual).\\[4pt]
$\operatorname{Int}([1,2]\cup\{3\}) = (1,2)$, \quad 
$\overline{[1,2]\cup\{3\}} = [1,2]\cup\{3\}$, \quad 
$Fr([1,2]\cup\{3\}) = \{1,2,3\}.$
\end{enumerate}

\subsection*{Ejercicio 6}
Determinar el interior, la frontera y la clausura de los siguientes conjuntos.

\begin{enumerate}
\item $\mathbb{Q}$ en $\mathbb{R}$ (topología usual).\\[4pt]
$\operatorname{Int}(\mathbb{Q}) = \varnothing$, \quad 
$\overline{\mathbb{Q}} = \mathbb{R}$, \quad 
$Fr(\mathbb{Q}) = \mathbb{R}$.

\item $(\mathbb{R}\setminus\mathbb{Z}) \times (\mathbb{R}\setminus\mathbb{Q})$ 
como subconjunto de $\mathbb{R}^2$ (topología usual).\\[4pt]
$\operatorname{Int} = \varnothing$ (pues $\mathbb{R}\setminus\mathbb{Q}$ no tiene interior),\\
$\overline{A} = \mathbb{R}^2$ \\
$Fr(A) = \mathbb{R}^2.$
\end{enumerate}

\section*{Ejercicio 7}
\begin{enumerate}

\item  $\displaystyle \operatorname{Fr}(A)=\overline{A}\cap\overline{A^c}$

\emph{Verdadero.} Si $x\in\operatorname{Fr}(A)$, todo abierto de $x$ corta $A$ y $A^c$,
luego $x\in\overline{A}$ y $x\in\overline{A^c}$. Recíprocamente,
si $x\in\overline{A}\cap\overline{A^c}$, todo abierto de $x$ corta ambos conjuntos,
así que $x\notin \operatorname{Int}(A)$ ni en $\operatorname{Int}(A^c)$ y, por tanto,
$x\in\operatorname{Fr}(A)$. También: 
\(\operatorname{Fr}(A)=\overline{A}\setminus \operatorname{Int}(A)=
\overline{A}\cap\overline{A^c}\).

\item  $\displaystyle \operatorname{Int}(A\cup B)=\operatorname{Int}(A)\cup\operatorname{Int}(B)$

\emph{Falso en general}. Siempre vale la inclusión
\(\operatorname{Int}(A)\cup\operatorname{Int}(B)\subseteq \operatorname{Int}(A\cup B)\),
pero la igualdad puede fallar. Contraejemplo en \(\mathbb{R}\) usual:
\(A=\mathbb{Q}\), \(B=\mathbb{R}\setminus\mathbb{Q}\).
Entonces \(A\cup B=\mathbb{R}\) y \(\operatorname{Int}(A\cup B)=\mathbb{R}\),
mientras que \(\operatorname{Int}(A)=\operatorname{Int}(B)=\varnothing\).

\item  $\displaystyle \operatorname{Int}(A\cap B)=\operatorname{Int}(A)\cap\operatorname{Int}(B)$

\emph{Verdadero}. Vía dualidad con clausura:
\[
\operatorname{Int}(A)=X\setminus\overline{A^c}.
\]
Así,
\[
\operatorname{Int}(A\cap B)=X\setminus\overline{(A\cap B)^c}
= X\setminus\overline{A^c\cup B^c}
= X\setminus\big(\overline{A^c}\cup\overline{B^c}\big)
= \big(X\setminus\overline{A^c}\big)\cap\big(X\setminus\overline{B^c}\big)
= \operatorname{Int}(A)\cap\operatorname{Int}(B).
\]

\item  $\displaystyle \operatorname{Int}(\operatorname{Fr}(A))=\varnothing$

\emph{Falso en general}. En \(\mathbb{R}\) con la topología usual, si \(A=\mathbb{Q}\),
entonces \(\overline{A}=\mathbb{R}\) y \(\operatorname{Int}(A)=\varnothing\),
así que \(\operatorname{Fr}(A)=\mathbb{R}\) y, por tanto,
\(\operatorname{Int}(\operatorname{Fr}(A))=\mathbb{R}\neq\varnothing\).
(Sí puede ocurrir que sea vacía para muchos \(A\), pero no es una verdad general.)
\end{enumerate}

\end{document}