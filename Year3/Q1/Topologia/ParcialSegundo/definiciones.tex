\documentclass[12pt]{article}
\usepackage[a4paper,margin=2.5cm]{geometry}
\usepackage{amsmath, amsfonts, amssymb, amsthm}

\title{Definiciones}
\author{Juan Rodríguez}
\date{}

\begin{document}
\maketitle

\section*{Conexidad}

\subsection*{Espacio conexo}

Un espacio topológico \(X\) se dice \textbf{conexo} si no existen dos abiertos disjuntos y no vacíos \(U,V\subset X\) tales que
\[
X=U\cup V.
\]

En caso de existir tal descomposición, se dice que \(X\) admite una \textbf{separación}.

Intuitivamente, un espacio es conexo si está hecho de una sola pieza, es decir, no puede dividirse en dos partes abiertas separadas.

\subsection*{Espacio conexo por caminos}

Un espacio topológico \(X\) se dice \textbf{conexo por caminos} si para cualquier par de puntos \(x,y\in X\) existe un camino continuo
\[
f:[0,1]\to X
\]
tal que
\[
f(0)=x,\qquad f(1)=y.
\]

\subsection*{Espacio localmente conexo}

Un espacio topológico \(X\) se dice \textbf{localmente conexo} si para todo punto \(x\in X\) y todo entorno abierto \(U\) de \(x\), existe un subentorno abierto \(V\) tal que
\[
x\in V\subset U
\]
y \(V\) es conexo.

\subsection*{Espacio localmente conexo por caminos}

Un espacio topológico \(X\) se dice \textbf{localmente conexo por caminos} si para todo punto \(x\in X\) y todo entorno abierto \(U\) de \(x\), existe un subentorno abierto \(V\) tal que
\[
x\in V\subset U
\]
y \(V\) es conexo por caminos.

\subsection*{Espacio simplemente conexo}

Un espacio topológico \(X\) se dice \textbf{simplemente conexo} si es conexo por caminos y su grupo fundamental es trivial, es decir,
\[
\pi_1(X)=\{e\}.
\]

\subsection*{Componente conexa}

Sea \(X\) un espacio topológico.  
Una \textbf{componente conexa} de \(X\) es un subconjunto \(C\subset X\) que es conexo tal que,
\[
\forall Z: C \subset Z \subset X,\quad Z \text{ no es conexo }
\]

Dicho de otro modo, una componente conexa es la \emph{máxima parte conexa} de \(X\) que contiene a un punto dado.

\section*{Espacio compacto}

Un espacio topológico \(X\) se dice \textbf{compacto} si todo recubrimiento abierto de \(X\) admite un subrecubrimiento finito.

\subsection*{Recubrimiento abierto}

Un \textbf{recubrimiento abierto} de \(X\) es una familia de abiertos \(\{U_i\}_{i\in I}\) tal que
\[
X=\bigcup_{i\in I} U_i.
\]

Un \textbf{subrecubrimiento finito} es una subfamilia finita
\[
U_{i_1},\dots,U_{i_n}
\]
tal que
\[
X=U_{i_1}\cup\cdots\cup U_{i_n}.
\]

\section*{Característica de Euler}

Sea \(G\) un grafo contenido en una superficie.  
La \textbf{característica de Euler} se define como
\[
\chi = V - A + F,
\]
donde:
\begin{itemize}
\item \(V\) es el número de vértices,
\item \(A\) es el número de aristas,
\item \(F\) es el número de zonas cerradas (caras).
\end{itemize}

\subsection*{Triangulación}

Una \textbf{triangulación} de una superficie es una descomposición de la misma en triángulos topológicos (posiblemente con lados curvos), de manera que dos triángulos cualesquiera se intersecan en una arista o vértice.

\section*{Grupo fundamental}

\subsection*{Lazo}

Sea \(X\) un espacio topológico y sea \(x_0\in X\).  
Un \textbf{lazo con base en \(x_0\)} es una aplicación continua
\[
\alpha:[0,1]\to X
\]
tal que
\[
\alpha(0)=\alpha(1)=x_0.
\]

\subsection*{Suma de lazos}

Sean \(f,g:[0,1]\to X\) dos lazos con el mismo punto base \(x_0\).  
La \textbf{suma de lazos} \(f+g\) se define como el lazo
\[
(f+g)(t)=
\begin{cases}
f(2t), & 0\le t\le \frac{1}{2},\\[4pt]
g(2t-1), & \frac{1}{2}<t\le 1.
\end{cases}
\]

Esta definición puede generalizarse reparametrizando los lazos según su longitud.

\subsection*{Lazo neutro}

El \textbf{lazo neutro} es el lazo constante
\[
e:[0,1]\to X,\qquad e(t)=x_0\ \ \forall t\in[0,1].
\]

Este lazo actúa como elemento identidad para la suma de lazos.

\subsection*{Lazo inverso}

Dado un lazo \(f:[0,1]\to X\) con base en \(x_0\), su \textbf{lazo inverso} se define como
\[
f^{-1}(t)=f(1-t), \qquad t\in[0,1].
\]

\subsection*{Homotopía de lazos}

Sean \(\alpha,\beta:[0,1]\to X\) dos lazos con base en \(x_0\).  
Se dice que son \textbf{homotópicos} si existe una aplicación continua
\[
h:[0,1]\times[0,1]\to X
\]
tal que
\[
h(t,0)=\alpha(t), \qquad h(t,1)=\beta(t),
\]
y además
\[
h(0,s)=h(1,s)=x_0 \quad \forall s\in[0,1].
\]

\subsection*{Lazos equivalentes}

Dos lazos se dicen \textbf{equivalentes} si son homotópicos fijando el punto base.

Esta relación define una relación de equivalencia en el conjunto de lazos con base en \(x_0\).

\subsection*{Grupo fundamental}

El \textbf{grupo fundamental} de \(X\) con base en \(x_0\), también llamado \textbf{grupo de Poincaré}, es el conjunto de clases de equivalencia de lazos con base en \(x_0\), con la operación inducida por la suma de lazos:
\[
\pi_1(X,x_0).
\]

\subsection*{Grupo cociente}

Sea \(A\) un conjunto y sea \(\sim\) una relación de equivalencia en \(A\).
El \textbf{conjunto cociente} es el conjunto de clases de equivalencia
\[
A/\!\sim \;=\;\{[a]\mid a\in A\}.
\]

Si además \(A\) está dotado de una operación binaria y la relación de equivalencia
\(\sim\) es compatible con dicha operación, entonces el conjunto cociente
hereda de forma natural una estructura de grupo, llamada \textbf{grupo cociente}.

\subsection*{Grupo fundamental como grupo cociente}

Sea \(X\) un espacio topológico y \(x_0\in X\).  
Sea \(\Omega(X,x_0)\) el conjunto de todos los lazos con base en \(x_0\).

Consideramos en \(\Omega(X,x_0)\) la relación de equivalencia dada por la homotopía
fijando el punto base.

El \textbf{grupo fundamental} de \(X\) con base en \(x_0\) se define como el grupo
del conjunto cociente
\[
\pi_1(X,x_0)
\;=\;
\Omega(X,x_0)\big/\sim,
\]
es decir, el conjunto de clases de equivalencia de lazos basados en \(x_0\)
módulo homotopía, con la operación inducida por la suma de lazos.

\end{document}