\documentclass[12pt]{article}
\usepackage[a4paper,margin=2.5cm]{geometry}
\usepackage{amsmath, amsfonts, amssymb, amsthm}

\title{Contra Ejemplos}
\author{Juan Rodríguez}
\date{}

\begin{document}
\maketitle
\section*{Un espacio localmente conexo no tiene por qué ser conexo}

\textbf{Ejemplo:}
\[
X = (0,1) \cup (1,2)
\]

El conjunto \(X\) no es conexo, ya que
\[
X = (0,1) \cup (1,2)
\]
es una separación en dos abiertos disjuntos y no vacíos.

Sin embargo, \(X\) es localmente conexo, pues para cualquier punto \(x \in X\) existe un entorno abierto contenido en \((0,1)\) o en \((1,2)\), respectivamente, y dichos intervalos son conexos.

Por tanto, un espacio localmente conexo no tiene por qué ser conexo.

\section*{Si la clausura es conexa, el conjunto no tiene por qué serlo}

\textbf{Ejemplo:}
\[
X = (0,1) \cup (1,2)
\]

El conjunto \(X\) no es conexo, ya que puede escribirse como la unión de dos abiertos disjuntos y no vacíos:
\[
X = (0,1) \cup (1,2).
\]

Sin embargo, su clausura en \(\mathbb{R}\) es
\[
\overline{X} = [0,1] \cup [1,2] = [0,2],
\]
que es un intervalo cerrado y acotado, por tanto, conexo en la topología canónica.

Por tanto, que la clausura de un conjunto sea conexa no implica que el conjunto lo sea.

\section*{Un espacio conexo no tiene por qué ser localmente conexo}

\textbf{Ejemplo:} el peine con el punto \((0,1)\).

\subsection*{El peine es conexo}

El peine es conexo porque puede expresarse como la unión de conjuntos conexos con puntos en común.

Cada púa
\[
\{(x,y)\mid x=\tfrac{1}{n},\; y\in[0,1]\}
\]
es homeomorfa al intervalo \([0,1]\), luego es conexa, y todas intersectan al mango
\[
\{(x,0)\mid x\in[0,1]\},
\]
que también es conexo.

Por el teorema de la unión de conexos con intersección no vacía, el peine es conexo.

\subsection*{No es localmente conexo}

El peine no es localmente conexo en el punto \((0,1)\).

En cualquier entorno abierto de \((0,1)\) aparecen puntos de la forma \((1/n,1)\), que pertenecen a componentes conexas distintas dentro de dicho entorno.

Por tanto, ningún entorno de \((0,1)\) contiene un subentorno abierto conexo que lo incluya, y el espacio no es localmente conexo.

\section*{Un conjunto conexo no tiene por qué ser conexo por caminos}

\textbf{Ejemplo:} el peine reducido.

\subsection*{Es conexo}

El peine completo es conexo, y el peine reducido se obtiene eliminando dos puntos, quedando como un conjunto intermedio entre un conexo y su clausura.

Además, puede expresarse como unión de conjuntos conexos con puntos en común, por lo que sigue siendo conexo.

\subsection*{No es conexo por caminos}

Supongamos que existe un camino continuo
\[
f:[0,1]\to P
\]
tal que
\[
f(0)=(0,1), \qquad f(1)=(0,0),
\]
donde \(P\) es el peine reducido.

Sea \(x=(0,1)\). Como \(f\) es continua, todo entorno abierto de \(x\) es imagen de un abierto de \([0,1]\). Tomemos un entorno abierto \(B\subset[0,1]\) conexo.

Entonces \(f(B)\) es conexo y contiene a \(x\). Sin embargo, en el peine reducido la única componente conexa que contiene a \(x\) es el propio punto \(x\), que es cerrado.

Por tanto, \(f^{-1}(\{x\})\) es abierto y cerrado en \([0,1]\), lo cual es imposible salvo que sea todo \([0,1]\).

Esto contradice la existencia del camino, luego el peine reducido no es conexo por caminos.

\section*{Dos espacios con el mismo grupo fundamental no siempre son homeomorfos}

\textbf{Ejemplo:} la esfera \(S^2\) y el plano \(\mathbb{R}^2\).

\subsection*{Tienen el mismo grupo fundamental}

El plano \(\mathbb{R}^2\) es simplemente conexo, ya que todos sus lazos son homotópicos al lazo trivial. Por tanto,
\[
\forall x_0 \in \mathbb{R}^2, \quad \pi_1(\mathbb{R}^2,x_0)=\{e\}.
\]

La esfera \(S^2\) también es simplemente conexa. En efecto, cualquier lazo puede contraerse a un punto, por lo que
\[
\forall x_0 \in S^2, \quad \pi_1(S^2,x_0)=\{e\}.
\]

\subsection*{No son homeomorfos}

La esfera \(S^2\) es compacta, al ser un subconjunto cerrado y acotado de \(\mathbb{R}^3\).

El plano \(\mathbb{R}^2\) no es compacto, ya que no es acotado.

Como la compacidad es un invariante topológico, \(S^2\) y \(\mathbb{R}^2\) no pueden ser homeomorfos.

\section*{Una función contractiva no siempre tiene un punto fijo si el espacio no es compacto}

\textbf{Ejemplo:}
\[
f:(0,1)\to(0,1), \qquad f(x)=\frac{x}{2}.
\]

Buscamos los puntos fijos de \(f\):
\[
\frac{x}{2}=x \iff x=0,
\]
pero \(0 \notin (0,1)\). Por tanto, \(f\) no tiene puntos fijos en \((0,1)\).

El punto fijo se encuentra en la frontera del espacio, fuera de \((0,1)\).

El Teorema del Punto Fijo de Banach garantiza la existencia de un punto fijo en espacios compactos como \([0,1]\), pero no es aplicable en \((0,1)\), ya que este espacio no es compacto.

En efecto, \((0,1)\) admite el recubrimiento abierto
\[
(0,1)=\bigcup_{n=2}^{\infty}\left(\frac{1}{n},1\right),
\]
del cual no se puede extraer un subrecubrimiento finito. Por tanto, \((0,1)\) no es compacto.

\section*{Un abierto puede ser compacto}

\textbf{Ejemplo:} un espacio con la topología cofinita.

En la topología cofinita, los abiertos son los subconjuntos cuyo complementario es finito.

Sea \(U\) un abierto cualquiera y sea \(\{U_i\}_{i\in I}\) un recubrimiento abierto de \(U\).  
Entonces los complementarios \(U_i^c\) son conjuntos finitos y
\[
U^c = \bigcap_{i\in I} U_i^c.
\]

Como la intersección de conjuntos finitos se estabiliza en una intersección finita, existe un subíndice finito \(i_1,\dots,i_n\) tal que
\[
U^c = U_{i_1}^c \cap \cdots \cap U_{i_n}^c.
\]

Tomando complementarios, se obtiene un subrecubrimiento finito de \(U\).

Por tanto, todo abierto en la topología cofinita es compacto.

\section*{Un espacio cerrado y acotado no siempre es compacto}

\textbf{Ejemplo:} el intervalo \([0,1]\) con la topología de Sorgenfrey.

El conjunto \([0,1]\) es cerrado y acotado en la recta real.

Consideremos el recubrimiento abierto en la topología de Sorgenfrey
\[
[0,1]=\bigcup_{n=1}^{\infty}\left[1-\frac{1}{n},\,1-\frac{1}{n+1}\right)\;\cup\;[1,2).
\]

Todos los conjuntos del recubrimiento son abiertos en Sorgenfrey y son disjuntos entre sí, por lo que ningún subrecubrimiento finito puede cubrir \([0,1]\).

Por tanto, \([0,1]\) no es compacto en la topología de Sorgenfrey, a pesar de ser cerrado y acotado.

\section*{Dos espacios con la misma característica de Euler no siempre son homeomorfos}

\textbf{Ejemplo:} el disco y el plano proyectivo.

La característica de Euler de una superficie se define como
\[
\chi = V - A + D,
\]
donde \(V\) es el número de vértices, \(A\) el de aristas y \(D\) el de discos poligonales de una triangulación.

Para el disco, considerando una triangulación simple, se obtiene
\[
\chi(\text{disco}) = 1 - 1 + 1 = 1.
\]

Para el plano proyectivo, a partir de una triangulación adecuada, se obtiene
\[
\chi(\mathbb{RP}^2) = 2 - 2 + 1 = 1.
\]

Sin embargo, el disco es una superficie orientable con borde, mientras que el plano proyectivo es no orientable y no tiene borde.

Como la orientación y la presencia de borde son invariantes topológicos, el disco y el plano proyectivo no son homeomorfos, a pesar de tener la misma característica de Euler.

\end{document}