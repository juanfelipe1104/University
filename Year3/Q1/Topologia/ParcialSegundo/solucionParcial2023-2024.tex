\documentclass[12pt]{article}
\usepackage[a4paper,margin=2.5cm]{geometry}
\usepackage{amsmath, amsfonts, amssymb, amsthm}

\title{Solución del Parcial 2023-2024}
\author{Juan Rodríguez}
\date{}

\begin{document}
\maketitle
\section*{Ejercicio 1}

\subsection*{a) \([0,2]\times\{0\}\) con la topología del orden en \(\mathbb{R}^2\)}

Sea
\[
X=[0,2]\times\{0\}\subset\mathbb{R}^2
\]

En \(X\), podemos presentar la separación:
\[
X = ((0,0),(1,0'5)) \cup ((1,0'5),(2,0))
\]
Por tanto, \(X\) no es conexo lo que implica que tampoco lo es por caminos.
\[
\boxed{[0,2]\times\{0\}\ \text{ no es conexo ni conexo por caminos}}
\]

\subsection*{b) \(\{0\}\times[0,2]\) con la topología del orden en \(\mathbb{R}^2\)}

Sea
\[
Y=\{0\}\times[0,2]\subset\mathbb{R}^2
\]

En \(Y\), el orden lexicográfico queda determinado por la segunda coordenada:
\[
(0,y) < (0,y') \iff y<y'.
\]
Por tanto, \(Y\) con la topología del orden es homeomorfo al intervalo \([0,2]\) con la topología canónica.

Luego \(Y\) es conexo y conexo por caminos.

\[
\boxed{\{0\}\times[0,2]\ \text{ es conexo y conexo por caminos}}
\]

\subsection*{c) \(P=\{x=0\}\cup\{y=\ln x\}\)}

Sea
\[
P = L \cup G,
\quad
L=\{(0,y): y\in\mathbb{R}\},
\quad
G=\{(x,\ln x): x>0\}.
\]

Observamos que \(L\cap G=\varnothing\). Además, $P = L \cup G$

Por lo tanto,
\[
P = L \cup G
\]
es una separación de \(P\) en dos abiertos disjuntos no vacíos. Por tanto, \(P\) no es conexo
y tampoco es conexo por caminos.

\[
\boxed{P\ \text{ no es conexo ni conexo por caminos}}
\]

\subsection*{d) \(A=\bigcup_{n\in\mathbb{N}} A_n\) y \(A\cup\{(1,0)\}\)}

Sea, para cada \(n\in\mathbb{N}\),
\[
A_n=\{(x,y)\in\mathbb{R}^2:0\le x\le 1,\ y=x/n\},
\qquad
A=\bigcup_{n\in\mathbb{N}} A_n.
\]

Cada \(A_n\) es la imagen continua de \([0,1]\) (por \(x\mapsto(x,x/n)\)), luego es conexo.
Además,
\[
(0,0)\in A_n \quad \forall n,
\]
así que la unión \(\bigcup_n A_n\) es conexa. Por tanto, \(A\) es conexo.

El punto \((1,0)\) pertenece a la clausura de \(A\), pues \((1,1/n)\in A_n\) y
\((1,1/n)\to(1,0)\).
Como la unión de un conexo con un punto de su clausura es conexa, se concluye que
\[
A\cup\{(1,0)\}\ \text{ es conexo}.
\]

Sin embargo, \(A\cup\{(1,0)\}\) no es conexo por caminos (ejemplo clásico de la ``escoba infinita''):
no existe un camino continuo contenido en \(A\cup\{(1,0)\}\) que una \((1,0)\) con \((0,0)\).

\[
\boxed{A\cup\{(1,0)\}\ \text{ es conexo pero no conexo por caminos}}
\]

\section*{Ejercicio 2}

Decidimos si los siguientes subconjuntos son compactos.

\subsection*{a) \([0,2]\cup[3,4]\)}

En \(\mathbb{R}\) con la topología usual, un conjunto es compacto si y solo si es cerrado y acotado
(Teorema de Heine--Borel).

El conjunto \([0,2]\cup[3,4]\) es cerrado (unión de cerrados) y acotado. Luego es compacto.

\[
\boxed{[0,2]\cup[3,4]\ \text{ es compacto}}
\]

\subsection*{b) \([0,2)\cup(2,4]\)}

El conjunto no es cerrado: el punto \(2\) es punto de acumulación, pero \(2\notin [0,2)\cup(2,4]\).
Luego no es compacto (en \(\mathbb{R}\), compacto \(\Rightarrow\) cerrado).

\[
\boxed{[0,2)\cup(2,4]\ \text{ no es compacto}}
\]

\subsection*{c) \(\{1/n:n\in\mathbb{N}\}\cup\{0\}\)}

Sea
\[
C=\left\{\frac1n:n\in\mathbb{N}\right\}\cup\{0\}.
\]
Es acotado. Además, su único punto de acumulación es \(0\), y \(0\in C\), luego \(C\) es cerrado.

Por Heine--Borel, \(C\) es compacto.

\[
\boxed{\left\{\frac1n:n\in\mathbb{N}\right\}\cup\{0\}\ \text{ es compacto}}
\]

\subsection*{d) \(\mathbb{R}\) con la topología cofinita}

Sea \((\mathbb{R},\tau_{cf})\) con la topología cofinita.
Recordamos que un abierto no vacío tiene complementario finito.

Sea \(\{U_i\}_{i\in I}\) un recubrimiento abierto de \(\mathbb{R}\).
Elige \(U_{i_0}\neq\varnothing\). Entonces \(\mathbb{R}\setminus U_{i_0}\) es finito, digamos
\(\mathbb{R}\setminus U_{i_0}=\{x_1,\dots,x_k\}\).
Como \(\{U_i\}\) recubre \(\mathbb{R}\), para cada \(x_j\) existe \(U_{i_j}\) tal que \(x_j\in U_{i_j}\).

Entonces
\[
U_{i_0}\cup U_{i_1}\cup\cdots\cup U_{i_k}=\mathbb{R},
\]
y hemos obtenido un subrecubrimiento finito. Por tanto, \(\mathbb{R}\) es compacto en la topología cofinita.

\[
\boxed{(\mathbb{R},\tau_{cf})\ \text{ es compacto}}
\]

\section*{Ejercicio 3}

Queremos hallar \(k\) tal que
\[
3K \;\cong\; 2T \,\#\, k\mathbb{RP}^2,
\]
donde \(K\) es la botella de Klein y \(T\) el toro.

Usamos la característica de Euler y la fórmula de suma conexa:
\[
\chi(X\#Y)=\chi(X)+\chi(Y)-2.
\]

Recordamos:
\[
\chi(T)=0,\qquad \chi(K)=0,\qquad \chi(\mathbb{RP}^2)=1.
\]

\subsection*{Cálculo de \(\chi(3K)\)}
\[
\chi(K\#K)=0+0-2=-2,
\qquad
\chi(K\#K\#K)=(-2)+0-2=-4.
\]
Luego
\[
\chi(3K)=-4.
\]

\subsection*{Cálculo de \(\chi(2T\# k\mathbb{RP}^2)\)}
Primero,
\[
\chi(2T)=\chi(T\#T)=0+0-2=-2.
\]
Además, para \(k\mathbb{RP}^2\) se tiene \(\chi(k\mathbb{RP}^2)=2-k\).
Entonces,
\[
\chi(2T\# k\mathbb{RP}^2)=\chi(2T)+\chi(k\mathbb{RP}^2)-2
= (-2) + (2-k) -2
= -(k+2).
\]

\subsection*{Igualación}
Como las superficies son homeomorfas, sus características de Euler coinciden:
\[
-4 = -(k+2)\quad\Rightarrow\quad k=2.
\]

\[
\boxed{k=2}
\]

\section*{Ejercicio 4}

Sea \(S\) una superficie compacta, sin borde, \textbf{orientable} y con
\[
\chi(S)=-2.
\]
Queremos identificar \(S\) y dar un diagrama poligonal.

\subsection*{Identificación por característica de Euler}

Si \(S\) es orientable, entonces \(S\cong \#^g T\) (suma conexa de \(g\) toros) y
\[
\chi(S)=2-2g.
\]
Imponiendo \(\chi(S)=-2\):
\[
2-2g=-2 \quad\Longrightarrow\quad 2g=4 \quad\Longrightarrow\quad g=2.
\]
Por tanto,
\[
\boxed{S \cong T\#T}
\]
(es decir, la superficie orientable de género \(2\), el ``doble toro'').

\subsection*{Diagrama poligonal}

Un modelo estándar para \(T\#T\) es un octógono con identificaciones
\[
a\,b\,a^{-1}\,b^{-1}\,c\,d\,c^{-1}\,d^{-1}.
\]

\section*{Ejercicio 5}

Dados los espacios:
\[
A=\{(x,y)\in\mathbb{R}^2: x=0\},\quad
B=S^2\setminus\{(4,0,0)\},
\]
\[
C=\{(x,y)\in\mathbb{R}^2: |x+y|<3\},\quad
D=\{(x,y)\in\mathbb{R}^2: x^2+y^2\le 9\},
\]
\[
E=\{(x,y)\in\mathbb{R}^2: x^2+y^2<9\},\quad
F=\{(x,y)\in\mathbb{R}^2: 0< x^2+y^2\le 9\},
\]
\[
G=F\setminus\{(1,1)\}.
\]

\subsection*{1) Clase homeomorfa a \(\mathbb{R}^2\)}

\begin{itemize}
\item \(B\cong \mathbb{R}^2\): por proyección estereográfica, la esfera menos un punto es homeomorfa al plano.
\item \(C\cong \mathbb{R}^2\): el cambio lineal \((u,v)=(x+y,x-y)\) es un homeomorfismo lineal de \(\mathbb{R}^2\) y lleva
\[
C=\{|x+y|<3\}
\]
al conjunto \((-3,3)\times\mathbb{R}\), que es homeomorfo a \(\mathbb{R}^2\) (por ejemplo con \(u\mapsto \tan(\frac{\pi}{6}u)\)).
\item \(E\cong \mathbb{R}^2\): el disco abierto es homeomorfo al plano (por ejemplo mediante una aplicación radial).
\end{itemize}

Por tanto,
\[
\boxed{B\cong C\cong E\cong \mathbb{R}^2.}
\]

\subsection*{2) \(A\) no es homeomorfo a \(B,C,E\)}

El conjunto \(A\) son los ejes de coordenadas (homeomorfo a \(\mathbb{R}\)).
Además, \(A\setminus\{0,0\}\) es disconexo (se separa en cuatro semirrectas),
mientras que \(\mathbb{R}^2\setminus\{q\}\) es conexo para cualquier \(q\in\mathbb{R}^2\).

Como el número de componentes conexas tras quitar un punto es invariante topológica, se concluye que
\[
\boxed{A \not\cong B,\quad A\not\cong C,\quad A\not\cong E.}
\]

\subsection*{3) \(D\) no es homeomorfo a los anteriores}

El conjunto \(D\) es compacto (cerrado y acotado en \(\mathbb{R}^2\)).
En cambio, \(A,B,C,E\) no son compactos.

Como la compacidad es invariante topológica,
\[
\boxed{D \not\cong A,\ B,\ C,\ E.}
\]

\subsection*{4) Clasificación de \(F\) y \(G\) mediante \(\pi_1\)}

El conjunto \(F\) es el disco cerrado de radio \(3\) con el origen eliminado.
Se retrae por deformación fuerte a la circunferencia \(S^1\) (proyección radial), luego
\[
\pi_1(F)\cong \pi_1(S^1)\cong \mathbb{Z}.
\]
Por tanto \(F\) no es homeomorfo a \(A,B,C,D,E\) (todos ellos son simplemente conexos salvo \(A\), o bien compactos).

El conjunto \(G=F\setminus\{(1,1)\}\) es el disco (con borde) con dos puntos eliminados.
Se retrae por deformación fuerte a un grafo con dos ciclos (equivalente a un ``ocho''), luego
\[
\pi_1(G)\cong \mathbb{Z}*\mathbb{Z}.
\]
En particular, \(\pi_1(G)\not\cong \pi_1(F)\), luego
\[
\boxed{G \not\cong F.}
\]

\subsection*{Conclusión}

Las clases de homeomorfismo quedan:

\[
\boxed{B\cong C\cong E,}
\]
y los demás no son homeomorfos entre sí:
\[
\boxed{A,\ D,\ F,\ G\ \text{ quedan cada uno en su propia clase}.}
\]

\section*{Ejercicio 6}

En ambos casos se nos da un polígono con lados identificados mediante colores y flechas.
Para identificar la superficie asociada contamos el número de clases de vértices \(V\),
aristas \(E\) y caras \(F\), calculamos la característica de Euler
\[
\chi = V - E + F,
\]
y estudiamos la orientabilidad.

\subsection*{Primer diagrama}

El polígono tiene \(8\) lados identificados en \(4\) pares, luego
\[
E = 4.
\]
Al seguir las identificaciones de los extremos de las aristas se obtienen
tres clases distintas de vértices, que denotamos por \(A,B,C\). Por tanto,
\[
V = 3.
\]
El interior del polígono constituye una única cara, así que
\[
F = 1.
\]

La característica de Euler es
\[
\chi = V - E + F = 3 - 4 + 1 = 0.
\]

Además, el patrón de identificaciones no preserva la orientación, por lo que
la superficie es no orientable.

En superficies no orientables se cumple
\[
\chi = 2 - k,
\]
donde \(k\) es el número de planos proyectivos en la suma conexa.
Imponiendo \(\chi = 0\), obtenemos
\[
0 = 2 - k \quad\Rightarrow\quad k = 2.
\]

Como \(\#^2 \mathbb{RP}^2\) es homeomorfa a la botella de Klein, concluimos que
\[
\boxed{\text{El primer diagrama representa una botella de Klein}.}
\]

\subsection*{Segundo diagrama}

El polígono tiene \(6\) lados identificados en \(3\) pares, luego
\[
E = 3.
\]
Siguiendo las identificaciones de los vértices se obtienen dos clases distintas
de vértices, por lo que
\[
V = 2.
\]
De nuevo, hay una única cara:
\[
F = 1.
\]

La característica de Euler es
\[
\chi = V - E + F = 2 - 3 + 1 = 0.
\]

En este caso, todas las identificaciones preservan la orientación, luego la
superficie es orientable. Para superficies orientables se cumple
\[
\chi = 2 - 2g,
\]
donde \(g\) es el género. Imponiendo \(\chi = 0\), obtenemos
\[
0 = 2 - 2g \quad\Rightarrow\quad g = 1.
\]

Por tanto,
\[
\boxed{\text{El segundo diagrama representa un toro}.}
\]

\section*{Ejercicio 7}

\subsection*{a) \(\pi_1(\mathbb{T}^2\times S^1)\)}

Usamos que el grupo fundamental del producto es el producto directo:
\[
\pi_1(X\times Y)\cong \pi_1(X)\times \pi_1(Y).
\]
Además,
\[
\pi_1(\mathbb{T}^2)\cong \mathbb{Z}^2,
\qquad
\pi_1(S^1)\cong \mathbb{Z}.
\]
Luego
\[
\pi_1(\mathbb{T}^2\times S^1)\cong \mathbb{Z}^2\times \mathbb{Z}\cong \mathbb{Z}^3.
\]
\[
\boxed{\pi_1(\mathbb{T}^2\times S^1)\cong \mathbb{Z}^3}
\]

\subsection*{b) \(\pi_1(\mathbb{R}^4)\)}

El espacio \(\mathbb{R}^4\) es contractible (se retrae por deformación fuerte a un punto),
por lo que su grupo fundamental es trivial:
\[
\boxed{\pi_1(\mathbb{R}^4)=\{e\}}.
\]

\subsection*{c) \(\pi_1(Y)\), donde \(Y=\{(x,y):|x|\le 3,\,|y|\le 3\}\cup\{(x,y):x^2=9\}\)}

Obsérvese que \(\{(x,y):x^2=9\}=\{x=3\}\cup\{x=-3\}\) está contenida en el borde del cuadrado
\(\{(x,y):|x|\le 3,\,|y|\le 3\}\). Por tanto,
\[
Y=\{(x,y):|x|\le 3,\,|y|\le 3\},
\]
que es un cuadrado cerrado, y por tanto contractible.

Luego,
\[
\boxed{\pi_1(Y)=\{e\}}.
\]

\subsection*{d) \(\pi_1(Z)\), donde \(Z=\{(x,y):x^2+y^2=9\}\cup\{(x,y):x^2=9\}\)}

El conjunto \(Z\) es la unión de la circunferencia de radio \(3\) con las dos rectas verticales
\(x=3\) y \(x=-3\). Estas rectas se pegan a la circunferencia en los puntos \((3,0)\) y \((-3,0)\)
y no añaden ciclos (son ramas contractibles).

Por tanto, \(Z\) se retrae por deformación fuerte a la circunferencia \(S^1\) (de radio \(3\)),
y en consecuencia:
\[
\pi_1(Z)\cong \pi_1(S^1)\cong \mathbb{Z}.
\]
\[
\boxed{\pi_1(Z)\cong \mathbb{Z}}.
\]

\end{document}