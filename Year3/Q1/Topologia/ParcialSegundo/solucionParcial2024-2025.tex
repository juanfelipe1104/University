\documentclass[12pt]{article}
\usepackage[a4paper,margin=2.5cm]{geometry}
\usepackage{amsmath, amsfonts, amssymb, amsthm}

\title{Solución del Parcial 2024-2025}
\author{Juan Rodríguez}
\date{}

\begin{document}
\maketitle

\section*{Ejercicio 1 (0,9 puntos)}
\textbf{Pregunta:} ¿Son conexos por caminos? ¿Son conexos? Si no, indica una separación.

%------------------------------------------------------------

\subsection*{1.a) \(\{(x,y)\in\mathbb{R}^2 \mid 1\le y\le 2\}\) con topología del orden}

Sea
\[
A=\{(x,y)\in\mathbb{R}^2\mid 1\le y\le 2\}.
\]

\subsubsection*{Conexidad}
Definimos
\[
U=\{(x,y)\in A \mid x<0\},\qquad 
V=\{(x,y)\in A \mid x>0\}.
\]
Entonces \(U\cap V=\varnothing\), \(U\neq\varnothing\), \(V\neq\varnothing\) y \(A=U\cup V\).

Además, \(U\) y \(V\) son abiertos en la topología inducida por el orden (son intersecciones con \(A\) de conjuntos del tipo \(\{(x,y)\mid x<0\}\) y \(\{(x,y)\mid x>0\}\), que son abiertos en el orden lexicográfico).

Por tanto, \(A\) no es conexo.

\subsubsection*{Conexidad por caminos}
Como \(A\) no es conexo, no puede ser conexo por caminos.

\[
\boxed{\text{\(A\) no es conexo ni conexo por caminos.}}
\]

%------------------------------------------------------------

\subsection*{1.b) \(\{(x,y)\in\mathbb{R}^2 \mid 1\le x\le 2\}\) con topología del orden}

Sea
\[
B=\{(x,y)\in\mathbb{R}^2\mid 1\le x\le 2\}.
\]

\subsubsection*{Conexidad (idea: continuo lineal)}
Con el orden lexicográfico restringido, \(B\) es un \emph{continuo lineal}:
\begin{itemize}
\item Es un conjunto linealmente ordenado.
\item Tiene la propiedad del supremo: todo subconjunto no vacío acotado superiormente en \(B\) tiene supremo en \(B\)
(porque al restringir \(x\) a \([1,2]\) no aparece ``salto'' en la primera coordenada, y dentro de cada fibra \(\{x\}\times\mathbb{R}\) se usa el orden usual en \(y\)).
\item No tiene puntos extremos ``aislados'' en el orden (densidad local en el sentido del orden).
\end{itemize}
Por el teorema estándar de espacios ordenados, todo continuo lineal es conexo.
Luego \(B\) es conexo.

\subsubsection*{No conexidad por caminos}
No es conexo por caminos.

En la topología del orden lexicográfico, cada subconjunto
\[
\{x\}\times\mathbb{R}
\]
se comporta como un ``día'' fijo, en el que la segunda coordenada \(y\)
representa las ``horas'' de ese día.

Un camino continuo sólo puede moverse dentro de un mismo día,
es decir, manteniendo constante la primera coordenada \(x\).

Por tanto, no existe un camino continuo que una dos puntos con distinta
primera coordenada, por ejemplo \((1,0)\) y \((2,0)\).

Luego el espacio no es conexo por caminos.

\[
\boxed{\text{\(B\) es conexo, pero no es conexo por caminos.}}
\]

%------------------------------------------------------------

\subsection*{1.c) \([1,2]\) con la topología de Sorgenfrey}

Sea \(C=[1,2]\) con la topología de Sorgenfrey (base \([a,b)\)).

La recta de Sorgenfrey es totalmente disconexa, luego \(C\) no es conexo (y por tanto tampoco es conexo por caminos).

Una separación explícita es:
\[
U=[1,3/2),\qquad V=[3/2,2].
\]
Aquí \(U\) es abierto en \(C\). Además \(V\) es abierto en \(C\) porque
\[
V=[3/2,\infty)\cap[1,2],
\]
y \([3/2,\infty)\) es abierto en Sorgenfrey.

Luego \(C=U\cup V\) con \(U\cap V=\varnothing\), \(U,V\neq\varnothing\).

\[
\boxed{\text{\([1,2]\) con Sorgenfrey no es conexo ni conexo por caminos.}}
\]

\section*{Ejercicio 2 (0,75 puntos)}

\textbf{Pregunta:} ¿Son compactos estos subconjuntos de \(\mathbb{R}\) con la
\textbf{topología de semirrectas derechas}?  
Si no lo son, indica un recubrimiento que no admite subrecubrimiento finito.

Recordamos que una base de la topología de semirrectas derechas es
\[
\mathcal{B}=\{[a,\infty)\mid a\in\mathbb{R}\}.
\]

%------------------------------------------------------------

\subsection*{2.a) \(A=[0,2]\)}

El conjunto \(A\) es compacto.

En efecto, \(A\) tiene mínimo \(0\).  
Dado cualquier recubrimiento abierto de \(A\), existe un abierto \([a,\infty)\)
tal que \(0\in[a,\infty)\).  
Este abierto cubre todo \(A\), luego existe un subrecubrimiento finito.

\[
\boxed{A \text{ es compacto}}
\]

%------------------------------------------------------------

\subsection*{2.b) \(B=[0,2)\)}

El conjunto \(B\) es compacto.

Al igual que en el caso anterior, \(B\) tiene mínimo \(0\), por lo que cualquier
abierto que contenga a \(0\) cubre todo \(B\).

\[
\boxed{B \text{ es compacto}}
\]

%------------------------------------------------------------

\subsection*{2.c) \(C=(0,2]\)}

El conjunto \(C\) no es compacto.

Consideramos el recubrimiento abierto
\[
C\subset \bigcup_{n=1}^{\infty}\left[\frac{1}{n},\infty\right).
\]

Este recubrimiento cubre \(C\), pero no admite subrecubrimiento finito.  
En efecto, si se toman finitísimos índices \(n_1,\dots,n_k\) y se define
\(N=\max\{n_1,\dots,n_k\}\), la unión finita es
\[
\left[\frac{1}{N},\infty\right),
\]
que no cubre los puntos de \(C\) tales que \(0<x<\frac{1}{N}\).

\[
\boxed{C \text{ no es compacto}}
\]

%------------------------------------------------------------

\subsection*{2.d) \(D=(-\infty,2)\)}

El conjunto \(D\) no es compacto.

Consideramos el recubrimiento abierto
\[
D\subset \bigcup_{n=1}^{\infty}[-n,\infty).
\]

Este recubrimiento no admite subrecubrimiento finito, pues si tomamos
\([-n_1,\infty),\dots,[-n_k,\infty)\) y \(N=\max\{n_1,\dots,n_k\}\), la unión es
\[
[-N,\infty),
\]
que no cubre los puntos de \(D\) menores que \(-N\).

\[
\boxed{D \text{ no es compacto}}
\]

%------------------------------------------------------------

\subsection*{2.e) \(E=(2,\infty)\)}

El conjunto \(E\) no es compacto.

Consideramos el recubrimiento abierto
\[
E\subset \bigcup_{n=1}^{\infty}\left[2+\frac{1}{n},\infty\right).
\]

Este recubrimiento cubre \(E\), pero no admite subrecubrimiento finito.  
En efecto, si tomamos finitísimos índices \(n_1,\dots,n_k\) y
\(N=\max\{n_1,\dots,n_k\}\), la unión finita es
\[
\left[2+\frac{1}{N},\infty\right),
\]
que no cubre los puntos \(x\) tales que \(2<x<2+\frac{1}{N}\), los cuales
pertenecen a \(E\).

\[
\boxed{E \text{ no es compacto}}
\]

%------------------------------------------------------------

\subsection*{Conclusión}

Son compactos:
\[
[0,2],\ [0,2).
\]

No son compactos:
\[
(0,2],\ (-\infty,2),\ (2,\infty).
\]

\section*{Ejercicio 3 (0,8 puntos)}

Sea
\[
A_m=\{(x,y)\in\mathbb{R}^2\mid y=\tfrac{m}{x}\},\quad m\in\mathbb{N},
\]
y
\[
A=\{(x,x)\mid 0\le x\le 1\}\ \cup\ \bigcup_{m\in\mathbb{N}} A_m.
\]

\subsection*{(1) Clausura \(\overline A\) y frontera}

Cada conjunto \(A_m\) es cerrado en \(\mathbb{R}^2\).
En efecto, si \((x_n,\tfrac{m}{x_n})\to (a,b)\) con \(a\neq 0\), entonces
\(x_n\to a\) y por continuidad \(b=\tfrac{m}{a}\), luego \((a,b)\in A_m\).
Si \(x_n\to 0\), entonces \(\tfrac{m}{x_n}\to \pm\infty\), por lo que no hay
límite en \(\mathbb{R}^2\). Por tanto, \(A_m\) no añade puntos límite finitos
fuera de sí mismo.

Además, el segmento \(\{(x,x)\mid 0\le x\le 1\}\) es compacto y por tanto cerrado.

Como \(A\) es unión (numerable) de cerrados y no aparecen nuevos puntos de acumulación
finitos al variar \(m\), se concluye que
\[
\overline A = A.
\]

Como \(A\) no contiene ningún abierto de \(\mathbb{R}^2\), se tiene
\(\operatorname{int}(A)=\varnothing\), y por tanto
\[
\partial A=\overline A\setminus \operatorname{int}(A)=A.
\]

\subsection*{(2) Conexidad, conexidad por caminos y conexidad local}

Cada \(A_m\) es conexo por caminos (es la gráfica continua \(x\mapsto \tfrac{m}{x}\)
sobre \((0,\infty)\), homeomorfa a \((0,\infty)\)). El segmento
\(\{(x,x)\mid 0\le x\le 1\}\) también es conexo por caminos.

Sin embargo, para \(m\neq n\) se tiene \(A_m\cap A_n=\varnothing\).
Además, el segmento \(\{(x,x)\mid 0\le x\le 1\}\) sólo intersecta a \(A_m\)
cuando \(x=x\) y \(x=\tfrac{m}{x}\), es decir \(x^2=m\). En \([0,1]\) esto ocurre
únicamente para \(m=1\) en el punto \((1,1)\).
Por tanto, \(A\) es unión de varias componentes conexas disjuntas, luego
\[
A \text{ no es conexo}.
\]
En consecuencia, \(A\) no es conexo por caminos.

Finalmente, \(A\) es localmente conexo: dado un punto \(p\in A\), existe un radio
\(\varepsilon>0\) tal que la intersección \(B(p,\varepsilon)\cap A\) es un arco
(conexo) contenido en una única curva \(A_m\) o en el segmento, salvo en \(p=(1,1)\),
donde la intersección es la unión de dos arcos que se cortan en \(p\), y sigue siendo
conexa. Por tanto, \(A\) es localmente conexo.

\section*{Ejercicio 4 (0,6 puntos)}

Queremos expresar
\[
T^2 \# K \# K
\]
como suma conexa de esferas y planos proyectivos.

Recordamos que la esfera es neutro:
\[
S^2 \# X \cong X,
\]
por lo que \(\#\,2\) esferas no cambia el resultado.

Además,
\[
K \cong \mathbb{RP}^2 \# \mathbb{RP}^2.
\]
Luego
\[
T^2 \# K \# K
\cong
T^2 \# (\mathbb{RP}^2 \# \mathbb{RP}^2) \# (\mathbb{RP}^2 \# \mathbb{RP}^2)
\cong
T^2 \# \#^4 \mathbb{RP}^2.
\]

Usamos la identidad
\[
T^2 \# \mathbb{RP}^2 \cong \#^3 \mathbb{RP}^2.
\]
Entonces
\[
T^2 \# \#^4 \mathbb{RP}^2
\cong
(\#^3 \mathbb{RP}^2)\# \#^3 \mathbb{RP}^2
=
\#^6 \mathbb{RP}^2.
\]

Por tanto, la superficie es homeomorfa a la suma conexa de \(6\) planos proyectivos.

\section*{Ejercicio 5}

Para decidir si los espacios son homeomorfos, utilizamos el siguiente invariante
topológico:

\begin{quote}
Si \(X\) y \(Y\) son homeomorfos, entonces al quitar un punto correspondiente,
los espacios resultantes tienen el mismo número de componentes conexas.
\end{quote}

\subsection*{Primer espacio}

Al eliminar un punto adecuado del primer espacio, el conjunto resultante se
desconecta en dos componentes conexas, cada una homeomorfa a una circunferencia.

\subsection*{Segundo y tercer espacio}

Si se elimina un punto análogo en el segundo y en el tercer espacio, ambos
siguen siendo conexos.

Por tanto, el primer espacio no es homeomorfo ni al segundo ni al tercero.

\subsection*{Comparación entre el segundo y el tercer espacio}

Eliminando otro punto distinto en ambos espacios, se obtiene que:
\begin{itemize}
\item el segundo espacio se desconecta en cuatro componentes conexas;
\item el tercer espacio se desconecta únicamente en dos componentes conexas.
\end{itemize}

Como el número de componentes conexas es distinto, el segundo y el tercer
espacio no son homeomorfos.

\subsection*{Conclusión}

Ninguno de los tres espacios es homeomorfo a otro.

\section*{Ejercicio 6 (1,2 puntos)}

Dados:
\[
A=\{(x,y)\in\mathbb{R}^2\mid 1\le x^2+y^2\le 4\},\qquad
B=\{(x,y,z)\in\mathbb{R}^3\mid x^2+y^2=9,\ 1\le z\le 4\},
\]
\[
C=\{(x,y)\in\mathbb{R}^2\mid x^2+y^2\le 4\},\qquad
D=\{(x,y)\in\mathbb{R}^2\mid x^2+y^2<4\}.
\]

\subsection*{(1) \(A\) y \(B\) son homeomorfos}

El conjunto \(A\) es un anillo cerrado, y es homeomorfo a \(S^1\times[0,1]\)
(vía coordenadas polares con radio en \([1,2]\)).
El conjunto \(B\) es la superficie lateral de un cilindro de radio \(3\) y altura finita,
y es homeomorfo a \(S^1\times[1,4]\).
Como \([1,4]\cong[0,1]\), se concluye:
\[
A \cong B.
\]

\subsubsection*{Homeomorfismo explícito entre \(A\) y \(B\)}

Recordemos:
\[
A=\{(x,y)\in\mathbb{R}^2\mid 1\le x^2+y^2\le 4\},\qquad
B=\{(X,Y,Z)\in\mathbb{R}^3\mid X^2+Y^2=9,\ 1\le Z\le 4\}.
\]

Definimos la aplicación
\[
f:A\longrightarrow B
\]
por
\[
f(x,y)=\left(
3\frac{x}{\sqrt{x^2+y^2}},\;
3\frac{y}{\sqrt{x^2+y^2}},\;
3\sqrt{x^2+y^2}-2
\right).
\]

Si \((x,y)\in A\), entonces \(r=\sqrt{x^2+y^2}\in[1,2]\). Se tiene
\[
\left(3\frac{x}{r}\right)^2+\left(3\frac{y}{r}\right)^2=9
\quad\text{y}\quad
3r-2\in[1,4],
\]
luego \(f(x,y)\in B\).

\subsection*{(2) \(C\) y \(D\) no son homeomorfos}

El conjunto \(C\) es compacto (cerrado y acotado en \(\mathbb{R}^2\)).
En cambio, \(D\) no es compacto (por ejemplo, no es cerrado: su clausura es \(C\)).
Como la compacidad es invariante topológica,
\[
C \not\cong D.
\]

\subsection*{(3) \(A\) y \(B\) no son homeomorfos ni a \(C\) ni a \(D\)}

Se tiene \(\pi_1(A)\cong\mathbb{Z}\) y \(\pi_1(B)\cong\mathbb{Z}\), al ser ambos
homeomorfos a \(S^1\times[0,1]\).
En cambio, \(C\) (disco cerrado) y \(D\) (disco abierto) son simplemente conexos,
luego \(\pi_1(C)=\pi_1(D)=\{e\}\).
Como el grupo fundamental es invariante topológica:
\[
A \not\cong C,\quad A\not\cong D,\quad B\not\cong C,\quad B\not\cong D.
\]

\subsection*{Conclusión}

Los únicos homeomorfismos son:
\[
\boxed{A\cong B}
\]
y no hay otros.

\section*{Ejercicio 7}

\textbf{Objetivo:} identificar la superficie de cada diagrama usando la característica de Euler
\[
\chi = V - A + F,
\]
donde \(V\) es el número de vértices, \(A\) el número de aristas y \(F\) el número de caras.

Recordamos:
\[
\chi(\#^g T^2)=2-2g \quad\text{(orientables)}, 
\qquad
\chi(\#^k \mathbb{RP}^2)=2-k \quad\text{(no orientables)}.
\]

\subsection*{Diagrama A}

Del diagrama se obtiene:
\[
V=1,\qquad A=4,\qquad F=1.
\]
Luego
\[
\chi(A)=V-A+F=1-4+1=-2.
\]
Además, el diagrama es \textbf{no orientable}. Por tanto debe ser de la forma \(\#^k\mathbb{RP}^2\), y
\[
2-k=-2 \;\Rightarrow\; k=4.
\]
Concluimos:
\[
\boxed{A \cong \#^4 \mathbb{RP}^2 \text{ (suma conexa de 4 planos proyectivos)}.}
\]

\subsection*{Diagrama B}

Del diagrama se obtiene:
\[
V=1,\qquad A=3,\qquad F=1.
\]
Luego
\[
\chi(B)=V-A+F=1-3+1=-1.
\]
Además, el diagrama es \textbf{no orientable}. Por tanto \(B\cong \#^k\mathbb{RP}^2\) y
\[
2-k=-1 \;\Rightarrow\; k=3.
\]
Concluimos:
\[
\boxed{B \cong \#^3 \mathbb{RP}^2 \text{ (suma conexa de 3 planos proyectivos)}.}
\]

\section*{Ejercicio 8}

\subsection*{a)}
Sea
\[
X=C_- \cup C_+ \cup S
\]
donde
\[
C_-=\{(x,y)\in\mathbb{R}^2:(x+2)^2+y^2=1\},\quad
C_+=\{(x,y)\in\mathbb{R}^2:(x-2)^2+y^2=1\},
\]
\[
S=\{(x,y)\in\mathbb{R}^2:|x|\le 1\}.
\]
El conjunto \(S\) es homeomorfo a \(\mathbb{R}^2\), luego es contractible.
Además, \(S\) se retrae por deformación fuerte al segmento
\[
I=[-1,1]\times\{0\}
\]
mediante \(H((x,y),t)=(x,(1-t)y)\). Manteniendo fijos los puntos de unión
\((-1,0)\) y \((1,0)\), el espacio \(X\) se retrae a
\[
G=C_- \cup I \cup C_+,
\]
que es un grafo con dos ciclos independientes. Por tanto,
\[
\pi_1(X)\cong \pi_1(G)\cong \mathbb{Z}*\mathbb{Z}.
\]

\subsection*{b)}
Usando que \(\pi_1(X\times Y)\cong \pi_1(X)\times\pi_1(Y)\),
\[
\pi_1(S^2\times S^1)\cong \pi_1(S^2)\times\pi_1(S^1)\cong \{e\}\times\mathbb{Z}\cong\mathbb{Z}.
\]

\subsection*{c)}
Sea
\[
Y=\mathbb{R}^3\setminus\{(x,y,z):x^2+y^2<1\}.
\]
Entonces
\[
Y\cong \bigl(\mathbb{R}^2\setminus\{(x,y):x^2+y^2<1\}\bigr)\times\mathbb{R}.
\]
El conjunto \(\mathbb{R}^2\setminus\{x^2+y^2<1\}\) se retrae por deformación fuerte
a la circunferencia \(S^1\) por proyección radial.

Consideramos el conjunto
\[
X=\mathbb{R}^2\setminus\{(x,y)\mid x^2+y^2<1\}.
\]

Definimos el retracto
\[
r:X\longrightarrow S^1,\qquad
r(x,y)=\left(\frac{x}{\sqrt{x^2+y^2}},\frac{y}{\sqrt{x^2+y^2}}\right).
\]

Además, definimos la homotopía
\[
H:X\times[0,1]\longrightarrow X
\]
por
\[
H((x,y),t)=\left((1-t)x+t\frac{x}{\sqrt{x^2+y^2}},\;
(1-t)y+t\frac{y}{\sqrt{x^2+y^2}}\right).
\]

Se verifica que:
\begin{itemize}
\item \(H(\cdot,0)=\mathrm{id}_X\),
\item \(H(\cdot,1)=r\),
\item \(H(p,t)=p\) para todo \(p\in S^1\) y todo \(t\in[0,1]\),
\item \(H((x,y),t)\in X\) para todo \((x,y)\in X\).
\end{itemize}

Por tanto, \(X\) se retrae por deformación fuerte a \(S^1\).

Luego \(Y\) se retrae a \(S^1\times\mathbb{R}\),
y por tanto
\[
\pi_1(Y)\cong \pi_1(S^1\times\mathbb{R})\cong \pi_1(S^1)\cong \mathbb{Z}.
\]

\end{document}