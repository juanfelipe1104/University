\documentclass[12pt]{article}
\usepackage[utf8]{inputenc}
\usepackage[spanish]{babel}
\usepackage{amsmath, amssymb}
\usepackage{geometry}
\geometry{margin=2.5cm}

\title{Ejercicios de Inferencia Estadística}
\author{}
\date{}

\begin{document}
\maketitle
\section*{Ejercicio 1}

Una compañía dedicada a la investigación sobre energía solar pretende averiguar el porcentaje de personas dispuestas a adquirir un vehículo con ese tipo de energía. Para ello decide realizar una encuesta en Madrid.

\begin{enumerate}
\item[(a)] Si los responsables de la compañía pretenden estimar dicho porcentaje con un error máximo del 3\% y una confianza del 99\%, ¿a cuántas personas como mínimo deberá pasarse la encuesta para asegurar las condiciones?

\item[(b)] Se decide finalmente realizar la encuesta a partir de una muestra de 500 personas y resulta que solo 63 respondieron que estarían dispuestos a comprar un vehículo solar. De manera independiente, una encuesta similar fue realizada en Barcelona y, de los 700 entrevistados, 79 se mostraron favorables a adquirir un vehículo de ese tipo.

¿Asegurarías al 95\% que el vehículo de energía solar tiene una mayor aceptación en una ciudad que en otra? Efectúa un contraste de hipótesis y explica la conclusión.

\item[(c)] Construir un intervalo de confianza $(1-\alpha=0{,}95)$ para la diferencia de proporciones.  
¿Aceptarías ahora que las proporciones de partidarios del vehículo solar son iguales en Madrid y Barcelona? Razona tus conclusiones.
\end{enumerate}

\section*{Ejercicio 2}

Se comenzó a administrar una nueva vacuna cuya efectividad se consideraría significativa si lograba reducir a la mitad la incidencia de la polio, que se estimaba inicialmente en 30 por cada 100\,000.

Se tomó una muestra de $200\,745$ niños a los que se administró la vacuna y, entre ellos, se observaron $33$ casos de polio.  
A otro grupo de $201\,229$ niños se les administró placebo y en este grupo hubo $110$ casos de polio.

\begin{enumerate}
\item[(a)] Construir un intervalo de confianza del $99\%$ para la diferencia de proporciones.
\item[(b)] ¿Qué puedes concluir sobre la efectividad de la vacuna?
\end{enumerate}

\section*{Ejercicio 3}
En un estudio sobre los préstamos concedidos por dos entidades financieras se toma una muestra aleatoria de $41$ préstamos de la primera entidad, observando un importe medio de $15\,000$ euros y una desviación típica de $9\,800$ euros.  
Al obtener los datos de una muestra aleatoria de $49$ préstamos de la segunda entidad se comprobó un importe medio de $13\,000$ euros y una desviación típica de $9\,300$ euros.

\begin{enumerate}
\item[(a)] ¿Proporcionan los datos evidencia $(\alpha = 0{,}01)$ de que el importe medio de los préstamos concedidos es significativamente mayor en la primera entidad?
\item[(b)] Construir un intervalo con un $99\%$ de confianza para el importe medio de los préstamos concedidos por la segunda entidad. Interpretar el resultado.
\item[(c)] ¿Cuántos créditos como mínimo habrá que analizar para estimar el importe medio de los créditos de la primera entidad con un error máximo de $300$ euros? (nivel de confianza: $0{,}99$)
\end{enumerate}
\end{document}