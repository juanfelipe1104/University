\documentclass[12pt]{article}
\usepackage[utf8]{inputenc}
\usepackage[spanish]{babel}
\usepackage{amsmath, amssymb}
\usepackage{geometry}
\geometry{margin=2.5cm}

\title{Soluciones de Ejercicios de Inferencia Estadística}
\author{}
\date{}
\begin{document}
\maketitle
\section*{Solución Ejercicio 1}

\subsection*{a) Tamaño muestral}

Partimos del intervalo de confianza $(1-\alpha)$ para una proporción:
\[
\left(
p - z_{1-\frac{\alpha}{2}} \sqrt{\frac{p(1-p)}{n}},
\;
p + z_{1-\frac{\alpha}{2}} \sqrt{\frac{p(1-p)}{n}}
\right)
\]

El error máximo de estimación es:
\[
E = z_{1-\frac{\alpha}{2}} \sqrt{\frac{\pi(1-\pi)}{n}}
\]

Como no disponemos de una estimación previa de $\pi$, utilizamos el valor máximo:
\[
\pi(1-\pi) \leq 0{,}25
\]

Imponemos que el error sea como máximo $0{,}03$:
\[
z_{1-\frac{\alpha}{2}} \sqrt{\frac{0{,}25}{n}} \leq 0{,}03
\]

Para un nivel de confianza del $99\%$:
\[
z_{1-\frac{\alpha}{2}} = 2{,}575
\]

Entonces:
\[
2{,}575 \sqrt{\frac{0{,}25}{n}} \leq 0{,}03
\quad \Longrightarrow \quad
n \geq 1841{,}84
\]

Por tanto, el tamaño mínimo de muestra es:
\[
\boxed{n = 1842}
\]

\subsection*{b) Contraste de hipótesis para dos proporciones}

Las proporciones muestrales son:
\[
p_1 = \frac{63}{500} = 0{,}126 \qquad
p_2 = \frac{79}{700} = 0{,}113
\]

Planteamos el contraste:
\[
\begin{cases}
H_0: \pi_1 - \pi_2 = 0 \\
H_A: \pi_1 - \pi_2 > 0
\end{cases}
\]

El estadístico de contraste es:
\[
Z_{\text{obs}} =
\frac{p_1 - p_2 - (\pi_1 - \pi_2)}
{\sqrt{\dfrac{p_1 q_1}{n_1} + \dfrac{p_2 q_2}{n_2}}}
\]

Sustituyendo valores:
\[
Z_{\text{obs}} =
\frac{0{,}126 - 0{,}113}
{\sqrt{\dfrac{0{,}126\cdot 0{,}874}{500} +
       \dfrac{0{,}113\cdot 0{,}887}{700}}}
= 0{,}68
\]

El p-valor es:
\[
p\text{-valor} = P(Z > 0{,}68) = 0{,}2483
\]

Como $p$-valor $> \alpha$ tanto para $\alpha=0{,}05$ como para $\alpha=0{,}01$, se concluye que:
\[
\boxed{\text{No se rechaza } H_0}
\]

No hay evidencia suficiente para afirmar que la aceptación sea mayor en Madrid que en Barcelona.

\subsection*{c) Intervalo de confianza para la diferencia de proporciones}

El intervalo de confianza $(1-\alpha)$ para $\pi_1 - \pi_2$ es:
\[
\left(
p_1 - p_2 - z_{1-\frac{\alpha}{2}}
\sqrt{\frac{p_1 q_1}{n_1} + \frac{p_2 q_2}{n_2}},
\;
p_1 - p_2 + z_{1-\frac{\alpha}{2}}
\sqrt{\frac{p_1 q_1}{n_1} + \frac{p_2 q_2}{n_2}}
\right)
\]

Para $\alpha=0{,}05$, $z_{1-\frac{\alpha}{2}}=1{,}96$:
\[
(0{,}013 - 1{,}96\cdot 0{,}019,\; 0{,}013 + 1{,}96\cdot 0{,}019)
= (0{,}013 \pm 0{,}037)
\]

Por tanto:
\[
\boxed{(-0{,}024,\; 0{,}05)}
\]

\begin{itemize}
\item Con un $95\%$ de confianza, la diferencia de proporciones está entre un $-2{,}4\%$ (mayor en Barcelona) y un $5\%$ (mayor en Madrid).
\item Como el valor $0$ pertenece al intervalo, no se detectan diferencias significativas entre las proporciones.
\end{itemize}

\[
\boxed{\text{Se acepta la igualdad de proporciones al nivel } \alpha=0{,}05}
\]

\subsection*{Solución Ejercicio 2}

Disponemos de las proporciones muestrales:
\[
p_2=\frac{33}{200\,745}=0{,}000164
\quad\text{(proporción muestral de niños vacunados que enfermaron de polio)}
\]
\[
p_1=\frac{110}{201\,229}=0{,}00055
\quad\text{(proporción muestral de niños con placebo que enfermaron de polio)}
\]

Queremos hacer inferencias sobre el parámetro $\pi_1-\pi_2$, la diferencia entre las proporciones poblacionales de niños que enfermaron (placebo frente a vacuna), que mide la efectividad de la vacuna.

\subsection*{a) Intervalo de confianza del 99\% para $\pi_1-\pi_2$}

Construimos el intervalo de confianza para la diferencia de proporciones:
\[
P\!\left(
a < 
\frac{(p_1-p_2)-(\pi_1-\pi_2)}
{\sqrt{\frac{p_1q_1}{n_1}+\frac{p_2q_2}{n_2}}}
< b
\right)=0{,}99
\]
En la tabla $N(0,1)$, para $0{,}99$:
\[
a=z_{0{,}005}=-2{,}575,
\qquad
b=z_{0{,}995}=2{,}575
\]

Luego:
\[
P\!\left(
-2{,}575<
\frac{(p_1-p_2)-(\pi_1-\pi_2)}
{\sqrt{\frac{p_1q_1}{n_1}+\frac{p_2q_2}{n_2}}}
<2{,}575
\right)=0{,}99
\]

Equivalentemente,
\[
(p_1-p_2)-2{,}575\sqrt{\frac{p_1q_1}{n_1}+\frac{p_2q_2}{n_2}}
<\pi_1-\pi_2<
(p_1-p_2)+2{,}575\sqrt{\frac{p_1q_1}{n_1}+\frac{p_2q_2}{n_2}}
\]

Sustituyendo:
\[
\pi_1-\pi_2 \in
\left(
0{,}000386 \pm 2{,}575\cdot 0{,}0000286
\right)
=
\left(0{,}000386 \pm 0{,}0000736\right)
\]

Por tanto, el intervalo es:
\[
\boxed{IC_{0{,}99}(\pi_1-\pi_2)=\left(0{,}00031,\;0{,}00046\right)}
\]

Interpretación:
\begin{itemize}
\item Con un $99\%$ de confianza, la proporción de niños \textbf{no vacunados} que contrajeron la polio estuvo entre $31$ y $46$ por cada $100\,000$ por encima de la proporción de niños \textbf{vacunados} que enfermaron.
\item Es decir, la vacuna consiguió reducir entre $31$ y $46$ casos por cada $100\,000$.
\end{itemize}

\subsection*{b) Conclusión sobre la efectividad: contraste para $\pi_2$}

Planteamos un contraste para comprobar si la vacuna logra que la proporción de niños vacunados que se contagian esté en $15$ por cada $100\,000$, es decir, $\pi_2=0{,}00015$.

\[
\begin{cases}
H_0:\ \pi_2 = 0{,}00015 & \text{(la vacuna reduce la incidencia a la mitad)}\\
H_A:\ \pi_2 > 0{,}00015 & \text{(la vacuna no logra el objetivo)}
\end{cases}
\]

El estadístico es:
\[
Z_{\text{obs}}=
\frac{p_2-\pi_2}{\sqrt{\frac{\pi_2(1-\pi_2)}{n_2}}}
=
\frac{0{,}000164-0{,}00015}
{\sqrt{\frac{0{,}00015\cdot 0{,}99985}{200\,745}}}
=0{,}0000273
\]

\[
p\text{-valor}=P(Z>0{,}0000273)=0{,}5>\alpha
\quad\Longrightarrow\quad
\boxed{\text{Aceptamos }H_0}
\]

Conclusión:
\begin{itemize}
\item Con un $1\%$ de margen de error, podemos admitir que la incidencia de polio en los niños vacunados fue de $15$ por cada $100\,000$,
\item por tanto, se redujo el valor inicial a la mitad y la vacuna se considera \textbf{efectiva}.
\end{itemize}

\section*{Solución Ejercicio 3}

Denotamos:
\[
\mu_1=\text{importe medio de todos los préstamos de la entidad A}
\]
\[
\mu_2=\text{importe medio de todos los préstamos de la entidad B}.
\]
Datos muestrales (en miles de euros):
\[
n_1=41,\quad \bar{x}=15,\quad s_1=9{,}8;
\qquad
n_2=49,\quad \bar{y}=13,\quad s_2=9{,}3.
\]

\subsection*{a) Contraste para la diferencia de medias}

Planteamos el contraste:
\[
\begin{cases}
H_0:\ \mu_1-\mu_2=0,\\
H_A:\ \mu_1-\mu_2>0.
\end{cases}
\]

Es un contraste sobre diferencia de medias con varianzas poblacionales desconocidas y tamaños muestrales grandes $(>40)$, por lo que usamos la aproximación normal:
\[
Z_{\text{obs}}=
\frac{(\bar{x}-\bar{y})-(\mu_1-\mu_2)}
{\sqrt{\frac{s_1^2}{n_1}+\frac{s_2^2}{n_2}}}
=
\frac{(15-13)-0}{\sqrt{\frac{96{,}04}{41}+\frac{86{,}49}{49}}}
=0{,}987.
\]

El p-valor (cola derecha) es:
\[
p\text{-valor}=P(Z>0{,}987)=0{,}1611>0{,}01.
\]

Por tanto,
\[
\boxed{\text{Aceptamos }H_0.}
\]
Conclusión: al $1\%$ no existen diferencias significativas entre los importes medios de los créditos concedidos por las dos entidades.

\subsection*{b) Intervalo de confianza del 99\% para $\mu_2$ (muestra grande, $\sigma$ desconocida)}

Para la entidad B, el intervalo $(1-\alpha)$ para la media es:
\[
\left(
\bar{y}-z_{1-\frac{\alpha}{2}}\frac{s_2}{\sqrt{n_2}},
\;
\bar{y}+z_{1-\frac{\alpha}{2}}\frac{s_2}{\sqrt{n_2}}
\right).
\]

Con $1-\alpha=0{,}99$, $z_{1-\frac{\alpha}{2}}=z_{0{,}995}=2{,}575$:
\[
\left(13\pm 2{,}575\frac{9{,}3}{\sqrt{49}}\right)
=
(13\pm 3{,}42)
=(9{,}58,\;16{,}42).
\]

Por tanto,
\[
\boxed{IC_{0{,}99}(\mu_2)=(9{,}58,\;16{,}42)\ \text{(miles de euros)}.}
\]

Interpretación:
\begin{itemize}
\item Con una confianza del $99\%$, el importe medio de los créditos de la entidad B está entre $9\,580$ y $16\,420$ euros.
\item Si tomamos muchas muestras de $49$ créditos y construimos el intervalo anterior, aproximadamente el $99\%$ de esos intervalos contendrán el valor real de $\mu_2$.
\end{itemize}

\subsubsection*{c) Tamaño muestral para estimar $\mu_1$ con error máximo 300 euros}

Queremos que el error máximo sea $E=0{,}3$ (miles de euros). Exigimos:
\[
z_{1-\frac{\alpha}{2}}\frac{s_1}{\sqrt{n}}\le 0{,}3.
\]

Con $z_{0{,}995}=2{,}575$ y $s_1=9{,}8$:
\[
2{,}575\frac{9{,}8}{\sqrt{n}}\le 0{,}3
\quad\Longrightarrow\quad
n\ge 7075{,}61.
\]

Luego, el tamaño mínimo es:
\[
\boxed{n=7076.}
\]

Es decir, hay que analizar al menos $7076$ créditos de la primera entidad para lograr ese margen de error con confianza $0{,}99$.

\end{document}