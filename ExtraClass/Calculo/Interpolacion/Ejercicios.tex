\documentclass[12pt]{article}
\usepackage[utf8]{inputenc}
\usepackage[spanish]{babel}
\usepackage{amsmath, amssymb}
\usepackage{geometry}
\geometry{margin=2.5cm}

\title{Ejercicios de Interpolación de Funciones (Taylor, Maclaurin y Lagrange)}
\author{}
\date{}

\begin{document}
\maketitle

\section*{Ejercicio 1}
Obtén el polinomio de Maclaurin de cuarto grado que permita aproximar el valor de
\[
\sqrt{1{,}2}.
\]
Calcula también una cota del error cometido en dicha aproximación.

\section*{Ejercicio 2}
Sea
\[
f(x)=x^{5}-3x^{4}-x^{3}+76x^{2}-78x+11.
\]
Usando el desarrollo de Taylor, expresa \(f(x)\) en potencias de \((x-1)\).

\section*{Ejercicio 3}
Dados los puntos
\[
(1,1),\quad (2,6),\quad (4,4),\quad (5,-1),
\]
considera el polinomio interpolador de Lagrange que pasa por ellos. Calcula el valor de dicho polinomio en
\[
x=3.
\]

\section*{Ejercicio 4}
Sabiendo que
\[
f(x)=\ln(1+ax)+b\,x\sin(cx)+d
\]
y que su polinomio de Maclaurin de tercer orden es
\[
P_3(x)=2x-3x^2+\frac{8}{3}\,x^3,
\]
completar:
\begin{enumerate}
  \item Determinar, de forma razonada, los valores de los parámetros \(a,b,c,d\in\mathbb{R}\).
  \item Obtener la expresión del resto de Lagrange asociado a \(P_3(x)\).
\end{enumerate}

\section*{Ejercicio 5}
Calcula el siguiente límite usando polinomios de Maclaurin:
\[
\lim_{x\to 0}\frac{\Big(\ln(1+x)+\frac{x^{2}}{2}-x\Big)^{10}}{\big(e^{x}-x-1\big)^{15}}.
\]

\end{document}
