\documentclass[12pt]{article}
\usepackage[utf8]{inputenc}
\usepackage[spanish]{babel}
\usepackage{amsmath, amssymb}
\usepackage{geometry}
\geometry{margin=2.5cm}

\title{Soluciones de Sucesiones y Series}
\author{}
\date{}

\begin{document}
\maketitle
\section*{Solución Ejercicio 1}
Consideramos el término general:
\[
u_n = \frac{n+2}{a^n(n+1)}.
\]

Para estudiar la convergencia en función de \(a\), aplicamos
el \textbf{criterio de comparación en el límite}.

\[
\lim_{n\to\infty} \frac{u_n}{\frac{1}{n^2}}
= 
\lim_{n\to\infty} \frac{\frac{n+2}{a^n(n+1)}}{\frac{1}{n^2}}
=
\lim_{n\to\infty} \frac{n^2(n+2)}{a^n(n+1)}
= 0
\quad\text{si } a>1.
\]

Puesto que \(\sum \frac{1}{n^2}\) converge, por comparación en el límite
\[
\sum_{n=1}^{\infty} u_n \quad \text{converge si } a>1.
\]

Ahora, comparamos con \(\sum \frac{1}{n}\) cuando \(a \leq 1\):

\[
\lim_{n\to\infty} \frac{u_n}{\frac{1}{n}}
=
\lim_{n\to\infty} \frac{\frac{n+2}{a^n(n+1)}}{\frac{1}{n}}
=
\lim_{n\to\infty} \frac{n(n+2)}{a^n(n+1)}
=
+\infty
\quad\text{si } a\le 1.
\]

Puesto que \(\sum \frac{1}{n}\) diverge, concluimos que también
\[
\sum_{n=1}^{\infty} u_n \quad \text{diverge si } a \le 1.
\]

\[
\boxed{
\text{La serie converge si } a>1
\quad\text{y diverge si } 0<a \le 1.
}
\]

\section*{Solución Ejercicio 2}
Sea
\[
a_n=\sqrt[3]{2}+\sqrt[6]{7}+\cdots+\sqrt[3n]{5n-3},
\qquad
b_n=7n+2.
\]
Entonces la expresión del límite es
\[
\lim_{n\to\infty}\frac{a_n}{b_n}.
\]

Observamos que \(a_n\) es creciente y \(b_n\) es creciente y no acotada, luego
podemos aplicar el \textbf{criterio de Stolz}:
\[
\lim_{n\to\infty}\frac{a_n}{b_n}
=
\lim_{n\to\infty}\frac{a_{n+1}-a_n}{b_{n+1}-b_n},
\]
si el límite de la derecha existe.

Calculamos:
\[
a_{n+1}-a_n = \sqrt[3(n+1)]{5(n+1)-3}
= \sqrt[3n+3]{5n+2},
\]
\[
b_{n+1}-b_n = 7.
\]

Por tanto
\[
L=\lim_{n\to\infty}\frac{a_n}{b_n}
=\lim_{n\to\infty}\frac{\sqrt[3n+3]{5n+2}}{7}
=\frac{1}{7}\,\lim_{n\to\infty}(5n+2)^{\frac{1}{3n+3}}.
\]

El límite interior es de tipo \(\infty^0\). Usamos la forma
\[
\lim f_n^{g_n} = e^{\lim g_n\ln f_n},
\]
si dicho límite existe. Tomamos
\[
f_n = 5n+2, \qquad g_n = \frac{1}{3n+3}.
\]

Entonces
\[
\lim_{n\to\infty}(5n+2)^{\frac{1}{3n+3}}
=
\exp\left(\lim_{n\to\infty}\frac{\ln(5n+2)}{3n+3}\right).
\]

El límite del exponente es
\[
\lim_{n\to\infty}\frac{\ln(5n+2)}{3n+3}=0,
\]
pues el numerador crece como \(\ln n\) y el denominador como \(n\).

Así,
\[
\lim_{n\to\infty}(5n+2)^{\frac{1}{3n+3}}
= e^{0}=1.
\]

Finalmente,
\[
L=\frac{1}{7}\cdot 1 = \frac{1}{7}.
\]

\[
\boxed{\displaystyle
\lim_{n\to\infty}
\frac{\sqrt[3]{2}+\sqrt[6]{7}+\cdots+\sqrt[3n]{5n-3}}{7n+2}
= \frac{1}{7}
}
\]

\section*{Solución Ejercicio 3}
Sea
\[
a_n=\frac{(n^{2}-3)^{3n}}{(6n^{6}+5n^{5}-n+1)^{n}}.
\]

Aplicamos el \textbf{criterio de la raíz}:
\[
L=\lim_{n\to\infty}\sqrt[n]{a_n}
= \lim_{n\to\infty}
\frac{\sqrt[n]{(n^{2}-3)^{3n}}}{\sqrt[n]{(6n^{6}+5n^{5}-n+1)^{n}}}
=
\lim_{n\to\infty}
\frac{(n^{2}-3)^{3}}{6n^{6}+5n^{5}-n+1}.
\]

Desarrollamos el numerador:
\[
(n^{2}-3)^{3}=n^{6}-9n^{4}+27n^{2}-27.
\]

Por tanto
\[
L
= \lim_{n\to\infty}
\frac{n^{6}-9n^{4}+27n^{2}-27}{6n^{6}+5n^{5}-n+1}.
\]

Dividimos numerador y denominador por \(n^{6}\):
\[
L = \lim_{n\to\infty}
\frac{1-\frac{9}{n^{2}}+\frac{27}{n^{4}}-\frac{27}{n^{6}}}
     {6+\frac{5}{n}-\frac{1}{n^{5}}+\frac{1}{n^{6}}}
= \frac{1}{6}.
\]

Como
\[
L=\frac{1}{6}<1,
\]
por el criterio de la raíz la serie converge.

\[
\boxed{\text{La serie es convergente.}}
\]

\section*{Solución Ejercicio 4}
\textbf{a)} Sea
\[
a_n = \frac{3^{n}n!}{n^{n}}.
\]
Aplicamos el \textbf{criterio del cociente}:
\[
\frac{a_{n+1}}{a_n}
=
\frac{3^{n+1}(n+1)!}{(n+1)^{n+1}}
\cdot
\frac{n^{n}}{3^{n}n!}
=
3\,\frac{(n+1)n^{n}}{(n+1)^{n+1}}
=
3\left(\frac{n}{n+1}\right)^{n}.
\]
Por tanto
\[
L=\lim_{n\to\infty}\frac{a_{n+1}}{a_n}
=
3\lim_{n\to\infty}\left(\frac{n}{n+1}\right)^{n}
= \frac{3}{e}.
\]
Como
\[
\frac{3}{e}>1,
\]
por el criterio del cociente la serie diverge.

\[
\boxed{\text{La serie a) es divergente (no converge ni absoluta ni condicionalmente).}}
\]

\textbf{b)} Sea ahora
\[
a_n = \frac{n!}{n^{n}}.
\]
De nuevo aplicamos el \textbf{criterio del cociente}:
\[
\frac{a_{n+1}}{a_n}
=
\frac{(n+1)!}{(n+1)^{n+1}}
\cdot
\frac{n^{n}}{n!}
=
\frac{(n+1)n^{n}}{(n+1)^{n+1}}
=
\left(\frac{n}{n+1}\right)^{n}.
\]
Entonces
\[
L=\lim_{n\to\infty}\frac{a_{n+1}}{a_n}
=
\lim_{n\to\infty}\left(\frac{n}{n+1}\right)^{n}
=\frac{1}{e}<1.
\]
Por el criterio del cociente la serie converge absolutamente
(al ser términos positivos).

\[
\boxed{\text{La serie b) es convergente y la convergencia es absoluta.}}
\]

\textbf{c)} Observamos que
\[
\cos(\pi n)=(-1)^{n},
\]
por lo que
\[
\frac{1}{n\cos(\pi n)}=\frac{(-1)^{n}}{n}.
\]
La serie se puede escribir como
\[
\sum_{n=1}^{\infty}\frac{1}{n\cos(\pi n)}
=\sum_{n=1}^{\infty}\frac{(-1)^{n}}{n},
\]
que es la \emph{serie armónica alternada}.

Sea \(a_n=\dfrac{(-1)^{n}}{n}\). Entonces
\[
\lim_{n\to\infty}a_n=0,
\qquad
|a_{n}|=\frac{1}{n},
\qquad
|a_{n+1}|=\frac{1}{n+1}<\frac{1}{n}=|a_n|.
\]
Es decir, los términos alternan de signo, su módulo es decreciente y tiende a
cero. Por el \textbf{criterio de Leibniz}, la serie es convergente.

Sin embargo,
\[
\sum_{n=1}^{\infty}|a_n|
=
\sum_{n=1}^{\infty}\frac{1}{n}
\]
es la serie armónica, que es divergente.

\[
\boxed{\text{La serie c) converge, pero sólo de forma condicional (no absoluta).}}
\]

\section*{Solución Ejercicio 5}
Definimos
\[
a_n = \sum_{k=1}^{n} k^{2}2^{k}, \qquad b_n = 2^{n}n^{2}.
\]
Entonces el límite pedido es
\[
L=\lim_{n\to\infty}\frac{a_n}{b_n}.
\]

Como \(b_n\) es creciente y tiende a infinito, podemos aplicar el
\textbf{criterio de Stolz}:
\[
L = \lim_{n\to\infty}\frac{a_{n+1}-a_n}{b_{n+1}-b_n}.
\]

Calculamos:
\[
a_{n+1}-a_n = (n+1)^2 2^{n+1},
\]
\[
b_{n+1}-b_n = 2^{n+1}(n+1)^2 - 2^{n}n^{2}
= 2^{n}\bigl(2(n+1)^2 - n^{2}\bigr).
\]

Por tanto
\[
L = \lim_{n\to\infty}
\frac{(n+1)^2 2^{n+1}}{2^{n}\bigl(2(n+1)^2 - n^{2}\bigr)}
= \lim_{n\to\infty}
\frac{2(n+1)^2}{2(n+1)^2 - n^{2}}.
\]

Desarrollamos numerador y denominador:
\[
2(n+1)^2 = 2(n^{2}+2n+1)=2n^{2}+4n+2,
\]
\[
2(n+1)^2 - n^{2} = (2n^{2}+4n+2) - n^{2}
= n^{2}+4n+2.
\]

Así,
\[
L = \lim_{n\to\infty}\frac{2n^{2}+4n+2}{n^{2}+4n+2}
= \frac{2}{1}=2.
\]

\[
\boxed{L=2.}
\]

\section*{Solución Ejercicio 6}
\begin{enumerate}
\item \textbf{Acotación.}

Como todos los términos se obtienen aplicando una raíz cuadrada,
se tiene \(a_n\ge 0\) para todo \(n\). Además \(a_1=0{,}5>0\), luego
la sucesión está acotada inferiormente por
\[
a_n \ge a_1 = 0{,}5, \qquad \forall n\in\mathbb{N}.
\]

Veamos ahora que está acotada superiormente por \(3\).  
Probamos por inducción que \(a_n<3\) para todo \(n\):

\begin{itemize}
  \item Caso base: 
  \[
  a_1=0{,}5<3.
  \]
  \item Paso inductivo: supongamos que para algún \(k\ge 1\) se cumple
  \(a_k<3\). Entonces
  \[
  a_{k+1} = \sqrt{a_k+6} < \sqrt{3+6} = \sqrt{9}=3.
  \]
\end{itemize}

Por inducción, \(a_n<3\) para todo \(n\).  
Por tanto, la sucesión está acotada:
\[
0{,}5 \le a_n < 3,\qquad \forall n.
\]

\item \textbf{Monotonía.}

Estudiamos el signo de \(a_n-a_{n-1}\). Para \(n\ge 2\),
\[
a_n-a_{n-1}
= \sqrt{a_{n-1}+6}-\sqrt{a_{n-2}+6}.
\]
Racionalizando:
\[
a_n-a_{n-1}
= \frac{(a_{n-1}+6)-(a_{n-2}+6)}
       {\sqrt{a_{n-1}+6}+\sqrt{a_{n-2}+6}}
= \frac{a_{n-1}-a_{n-2}}{\sqrt{a_{n-1}+6}+\sqrt{a_{n-2}+6}}.
\]
El denominador es siempre positivo, luego el signo de \(a_n-a_{n-1}\)
coincide con el de \(a_{n-1}-a_{n-2}\).

Calculamos el primer incremento:
\[
a_2-a_1 = \sqrt{a_1+6}-a_1 = \sqrt{0{,}5+6}-0{,}5 \approx 2{,}46-0{,}5>0.
\]
Por lo tanto \(a_2>a_1\), y como cada diferencia tiene el mismo signo
que la anterior, se deduce que
\[
a_n-a_{n-1}>0\quad\forall n\ge 2.
\]
Es decir, la sucesión es \textbf{estrictamente creciente}.

\item \textbf{Límite.}

Hemos probado que \((a_n)\) es creciente y está acotada superiormente,
por lo que es convergente. Sea
\[
\lim_{n\to\infty} a_n = L.
\]
Pasando al límite en la relación recurrente
\[
a_{n+1}=\sqrt{a_n+6},
\]
obtenemos
\[
L = \sqrt{L+6}.
\]
Elevando al cuadrado:
\[
L^2 = L+6
\quad\Longrightarrow\quad
L^2-L-6=0
\quad\Longrightarrow\quad
(L-3)(L+2)=0.
\]
De aquí \(L=3\) o \(L=-2\). Como todos los términos de la sucesión son
positivos, el límite debe ser positivo, luego
\[
L=3.
\]

\[
\boxed{\text{La sucesión es creciente, está acotada y converge a }3.}
\]
\end{enumerate}
\section*{Solución Ejercicio 7}
\begin{enumerate}
\item Sea \(a_n=\dfrac{n^{3}}{4^{n}}\). Aplicamos el criterio del cociente:
\[
L=\lim_{n\to\infty}\frac{a_{n+1}}{a_n}
 =\lim_{n\to\infty}\frac{(n+1)^{3}4^{n}}{4^{n+1}n^{3}}
 =\frac14\lim_{n\to\infty}\frac{(n+1)^{3}}{n^{3}}
 =\frac14<1.
\]
Por el criterio del cociente, la serie
\(\displaystyle\sum_{n=1}^{\infty}\frac{n^{3}}{4^{n}}\) es convergente.  
Como todos los términos son positivos, la convergencia es \textbf{absoluta}.

\item Consideramos
\[
a_n=\frac{\cos(n^{2})+2}{n\sqrt{n}}.
\]
Se tiene \(|\cos(n^{2})+2|\le 3\), luego
\[
|a_n|\le\frac{3}{n\sqrt{n}}=\frac{3}{n^{3/2}}.
\]
La serie de comparación
\(\displaystyle\sum_{n=1}^{\infty}\frac{3}{n^{3/2}}\) es convergente por Pringsheim con
\(p=\tfrac32>1\).  
Por el criterio de comparación, la serie
\(\displaystyle\sum_{n=1}^{\infty}\frac{\cos(n^{2})+2}{n\sqrt{n}}\) es
convergente y, además, lo es \textbf{absolutamente}.

\item Sea
\[
a_n=(-1)^{n}\sen\!\left(\frac1{\sqrt[3]{n}}\right).
\]
En primer lugar,
\[
\lim_{n\to\infty}\sen\!\left(\frac1{\sqrt[3]{n}}\right)=0,
\]
y los argumentos \(\dfrac1{\sqrt[3]{n}}\) son positivos y decrecientes a \(0\),
por lo que
\(\bigl|\sen(1/\sqrt[3]{n})\bigr|\) es también una sucesión decreciente.  
Luego, por el criterio de Leibniz, la serie alternada
\(\displaystyle\sum_{n=1}^{\infty}(-1)^{n}\sen\!\left(\frac1{\sqrt[3]{n}}\right)\)
es \textbf{convergente}.

Para estudiar la convergencia absoluta, observamos que
\[
\sen\!\left(\frac1{\sqrt[3]{n}}\right)\sim \frac1{\sqrt[3]{n}}
\quad\text{cuando }n\to\infty.
\]
Por tanto, la serie
\[
\sum_{n=1}^{\infty}\left|\sen\!\left(\frac1{\sqrt[3]{n}}\right)\right|
\]
se comporta como
\(\displaystyle\sum_{n=1}^{\infty}\frac1{n^{1/3}}\), que es divergente por Pringsheim con
\(p=\tfrac13<1\).  

En conclusión, la serie del apartado c) es \textbf{convergente
condicionalmente}.
\end{enumerate}
\section*{Solución Ejercicio 8}
\begin{enumerate}
\item Consideramos la serie
\[
\sum_{n=1}^{\infty}\frac{\cos^2(n)}{2^n}.
\]
Como \(0\le\cos^2(n)\le 1\) para todo \(n\in\mathbb{N}\), se tiene
\[
0\le\frac{\cos^2(n)}{2^n}\le\frac1{2^n}.
\]
La serie geométrica \(\displaystyle\sum_{n=1}^{\infty}\frac1{2^n}\) es convergente
(ratio \(1/2<1\)).  
Por el criterio de comparación, la serie
\(\displaystyle\sum_{n=1}^{\infty}\frac{\cos^2(n)}{2^n}\) es \textbf{convergente}
(y, de hecho, absolutamente convergente).

\item Ahora estudiamos
\[
\sum_{n=1}^{\infty}\frac{\sqrt[3]{n^4}}{\sqrt{n^3+4n+3}}.
\]
Aplicamos el criterio de Pringsheim para determinar el comportamiento de \(a_n\)
cuando \(n\to\infty\).
\[
\lim_{n\to\infty}a_n \cdot n^{1/6}
=\lim_{n\to\infty}\frac{\sqrt[3]{n^4}\,n^{1/6}}{\sqrt{n^3+4n+3}}
=1.
\]

Por Pringsheim, \(\alpha = \frac{1}{6} \leq 1\), luego la serie
\(\displaystyle\sum_{n=1}^{\infty}\frac{\sqrt[3]{n^4}}{\sqrt{n^3+4n+3}}\) es \textbf{divergente}.
\end{enumerate}

\section*{Solución Ejercicio 9}
\begin{enumerate}
\item 
\[
a_n=\frac{(n+1)^{n+1}}{n^n(2n+1)}
   =\left(\frac{n+1}{n}\right)^n\cdot\frac{n+1}{2n+1}.
\]

Por un lado,
\[
\lim_{n\to\infty}\frac{n+1}{2n+1}=\frac12.
\]

Por otro lado, el límite
\[
\lim_{n\to\infty}\left(\frac{n+1}{n}\right)^n
=\lim_{n\to\infty}\left(1+\frac1n\right)^n
\]
es de tipo $1^\infty$. Lo resolvemos con
\[
\left(1+\frac1n\right)^n = e^{A_n},\qquad
A_n = n\ln\!\left(1+\frac1n\right),
\]
y
\[
A=\lim_{n\to\infty}A_n
   =\lim_{n\to\infty}n\!\left(\frac1n+o\!\left(\frac1n\right)\right)=1,
\]
luego
\[
\lim_{n\to\infty}\left(1+\frac1n\right)^n=e^1=e.
\]

Por tanto,
\[
\lim_{n\to\infty}a_n
=\left(\lim_{n\to\infty}\left(\frac{n+1}{n}\right)^n\right)
 \left(\lim_{n\to\infty}\frac{n+1}{2n+1}\right)
=e\cdot\frac12=\frac e2.
\]

La sucesión es convergente y
\[
a_n \longrightarrow \frac e2.
\]

\item 
Consideramos
\[
a_n=\frac{n^2+(-1)^n n}{\sen(n)-n^2}.
\]

Dividimos numerador y denominador por $n^2$:
\[
a_n
=\frac{1+\dfrac{(-1)^n}{n}}{\dfrac{\sen(n)}{n^2}-1}.
\]

Como $\left|\sen(n)\right|\leq 1$ para todo $n$, se tiene
\[
\lim_{n\to\infty}\frac{\sen(n)}{n^2}=0,
\qquad
\lim_{n\to\infty}\frac{(-1)^n}{n}=0.
\]

Por tanto,
\[
\lim_{n\to\infty}a_n
=\frac{1+0}{0-1}=-1.
\]

Luego la sucesión es convergente y
\[
a_n \longrightarrow -1.
\]
\end{enumerate}
\end{document}