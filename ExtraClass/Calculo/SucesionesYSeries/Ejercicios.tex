\documentclass[12pt]{article}
\usepackage[utf8]{inputenc}
\usepackage[spanish]{babel}
\usepackage{amsmath, amssymb}
\usepackage{geometry}
\geometry{margin=2.5cm}

\title{Ejercicios de Sucesiones y Series}
\author{}
\date{}

\begin{document}
\maketitle
\section*{Ejercicio 1}
Determinar el carácter de la serie
\[
\sum_{n=1}^{\infty} \frac{n+2}{a^n(n+1)}
\quad \text{en función del parámetro } a \in \mathbb{R}, \; a>0.
\]

\section*{Ejercicio 2}
Calcular el siguiente límite utilizando el criterio de Stolz:
\[
\lim_{n\to\infty}
\frac{\sqrt[3]{2}+\sqrt[6]{7}+\cdots+\sqrt[3n]{5n-3}}{7n+2}.
\]

\section*{Ejercicio 3}
Determinar el carácter de la siguiente serie usando alguno de los criterios
vistos en clase:
\[
\sum_{n=1}^{\infty}
\frac{(n^{2}-3)^{3n}}{(6n^{6}+5n^{5}-n+1)^{n}}.
\]

\section*{Ejercicio 4}
Determinar si las series
\[
\text{a)}\quad \sum_{n=1}^{\infty}\frac{3^{n}n!}{n^{n}},
\qquad
\text{b)}\quad \sum_{n=1}^{\infty}\frac{n!}{n^{n}},
\qquad
\text{c)}\quad \sum_{n=1}^{\infty}\frac{1}{n\cos(\pi n)}
\]
son convergentes o divergentes y, adicionalmente, si son absoluta o condicionalmente
convergentes.

\section*{Ejercicio 5}
Calcular el siguiente límite utilizando el criterio de Stolz:
\[
\lim_{n\to\infty}
\frac{1^{2}2^{1}+2^{2}2^{2}+3^{2}2^{3}+\cdots+n^{2}2^{n}}{2^{n}n^{2}}.
\]

\section*{Ejercicio 6}
Dada la sucesión definida por
\[
a_{n+1}=\sqrt{a_n+6},\quad n\ge 1,\qquad a_1=0{,}5,
\]
completar los siguientes apartados:

\begin{enumerate}
  \item Demostrar que la sucesión está acotada tanto inferiormente 
        como superiormente, indicando claramente cuál es la cota inferior 
        y superior.
  \item Demostrar que la sucesión es monótona creciente o monótona decreciente.
  \item En caso de que la sucesión sea convergente, proporcionar su límite.
        En caso contrario, demostrar que es divergente.
\end{enumerate}

\section*{Ejercicio 7}
Determinar razonadamente si las siguientes series son convergentes o divergentes.
En el caso que proceda, indicar si son absoluta o condicionalmente convergentes.
\[
\text{a)}\ \sum_{n=1}^{\infty}\frac{n^{3}}{4^{n}},\qquad
\text{b)}\ \sum_{n=1}^{\infty}\frac{\cos(n^{2})+2}{n\sqrt{n}},\qquad
\text{c)}\ \sum_{n=1}^{\infty}(-1)^{n}\sen\!\left(\frac1{\sqrt[3]{n}}\right).
\]

\section*{Ejercicio 8}
Determina razonadamente si las siguientes series \(\displaystyle \sum_{n=1}^\infty a_n\)
son convergentes o divergentes, utilizando para ello alguno de los métodos o criterios
vistos en clase:

\[
\text{a)}\quad a_n=\frac{\cos^2(n)}{2^n},\qquad
\text{b)}\quad a_n=\frac{\sqrt[3]{n^4}}{\sqrt{n^3+4n+3}}.
\]

\section*{Ejercicio 9}
Determina razonadamente si las siguientes sucesiones $\{a_n\}$ son convergentes o
divergentes. En el caso de que sean convergentes, indica el valor al que convergen:
\begin{enumerate}
  \item $a_n=\dfrac{(n+1)^{n+1}}{n^n(2n+1)}$.
  \item $a_n=\dfrac{n^2+(-1)^n n}{\sen(n)-n^2}$.
\end{enumerate}
\end{document}