\documentclass[12pt]{article}
\usepackage[utf8]{inputenc}
\usepackage[spanish]{babel}
\usepackage{amsmath, amssymb}
\usepackage{geometry}
\geometry{margin=2.5cm}

\title{Ejercicios de Integrales Indefinidas y Definidas}
\author{}
\date{}

\begin{document}
\maketitle
\section*{Ejercicio 1}
Dada la función $f(x)=e^{2x}-1$, se define
\[
F(x)=\int_{1}^{x^2} f(t-1)\,dt \qquad \text{para todo } x\in[1,\infty).
\]
Calcular $F'(x)$ y determinar si esa derivada se anula en algún punto de su dominio.

\section*{Ejercicio 2}
Calcular el valor del parámetro $a\in\mathbb{R}$, con $a>0$, para que la recta $y=ax$ divida a la región acotada que limitan las curvas
\[
2y^2 = x
\qquad\text{y}\qquad
y = 2x^2,
\]
en dos partes, una de área doble que la otra.

\section*{Ejercicio 3}
Hallar el área de la región limitada por las curvas
\[
f(x)=\frac{1}{x}
\qquad\text{y}\qquad
g(x)=\frac{3x}{3x^2+6x+10},
\]
entre $x=1$ y $x=4$.

\section*{Ejercicio 4}
Calcular la siguiente integral:
\[
\int \frac{x\cos(x)-\sen(x)}{2x^2+\sen^2(x)}\,dx.
\]

\section*{Ejercicio 5}
Determinar el área geométrica encerrada por la gráfica de la función
\[
f(x)=x^{3}\arctan(x),
\]
el eje $X$ y las rectas $x=-1$ y $x=1$.

\section*{Ejercicio 6}
Calcular el volumen del sólido engendrado al girar en torno al eje $OX$ 
la superficie limitada por la curva
\[
f(x)=\frac{1}{(x-3)^2\sqrt{x+2}}
\]
y las rectas $x=0$ y $x=2$.

\section*{Ejercicio 7}
Calcular la integral definida
\[
\int_{e}^{e^{2}} \frac{Ln(Ln(x))}{x}\,dx.
\]

\section*{Ejercicio 8}
Una curva plana puede estar definida mediante ecuaciones cartesianas 
\((y=f(x))\) o mediante ecuaciones paramétricas \((x=x(t),\,y=y(t))\), donde \(t\) es el parámetro.
Sabiendo que en este último caso la longitud de una curva desde \(t=a\) hasta \(t=b\) viene dada por
\[
L=\int_a^b \sqrt{\left(\frac{dx(t)}{dt}\right)^2+\left(\frac{dy(t)}{dt}\right)^2}\,dt,
\]
calcular la longitud de la curva definida por las ecuaciones paramétricas
\[
x(t)=\frac{1}{3}t^3,\qquad y(t)=\frac{1}{2}t^2
\]
desde \(t=0\) hasta \(t=1\).

\section*{Ejercicio 9}
Calcula la integral
\[
\int_{\pi/4}^{\pi/2}\frac{2\sen(x)}{\cos^{2}(x)-3\cos(x)+2}\,dx
\]
y proporciona tanto la expresión más simplificada posible correspondiente a la solución
como su valor numérico

\section*{Ejercicio 10}
Hallar el volumen del sólido de revolución que se obtiene al hacer girar
la gráfica de $f(x)=-\sen x$ alrededor del eje $OX$ entre las abscisas $x=0$ y $x=2\pi$.

\section*{Ejercicio 11}
Se tiene una barra metálica cuya densidad lineal viene dada por
\[
\lambda(x)=2+\frac{x}{2}+\frac{1}{\sqrt{x+1}}\quad (\text{kg/m}),
\]
donde $x$ es la distancia desde uno de sus extremos, medida en metros.
Calcular la masa total de la barra si esta mide un total de $4$ metros.

\section*{Ejercicio 12}
Demostrar que la función
\[
F(x)=-1+\int_{0}^{x} e^{t^{2}}\,dt
\]
se anula en un único punto y acotar el intervalo en el que se encuentra dicho punto.
\end{document}