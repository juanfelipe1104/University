\documentclass[12pt]{article}
\usepackage[utf8]{inputenc}
\usepackage[spanish]{babel}
\usepackage{amsmath, amssymb}
\usepackage{geometry}
\geometry{margin=2.5cm}

\title{Soluciones de Integrales Indefinidas y Definidas}
\author{}
\date{}

\begin{document}
\maketitle
\section*{Solución Ejercicio 1}
Observamos que
\[
f(t-1)=e^{2(t-1)}-1 = e^{2t-2}-1.
\]
Por tanto,
\[
F(x)=\int_{1}^{x^2} \left(e^{2t-2}-1\right)dt.
\]

Aplicando el Teorema Fundamental del Cálculo junto a la regla de la cadena obtenemos
\[
F'(x) = \left(e^{2x^2-2}-1\right)\cdot 2x.
\]

Para estudiar dónde se anula la derivada resolvemos $F'(x)=0$:
\[
\left(e^{2x^2-2}-1\right)\cdot 2x = 0.
\]

Luego:
\[
2x=0 \quad \text{o} \quad e^{2x^2-2}-1=0.
\]

El primer caso no da soluciones en $[1,\infty)$, y en el segundo tenemos:
\[
e^{2x^2-2}=1 \quad \Longrightarrow\quad 2x^2-2=0 \quad \Longrightarrow\quad x=1.
\]

\[
\boxed{F'(x)=0 \ \text{únicamente en } x=1.}
\]

\section*{Solución Ejercicio 2}
Igualamos ambas curvas para hallar los extremos de la región:
\[
\sqrt{\frac{x}{2}} = 2x^2
\quad\Longrightarrow\quad
\frac{x}{2} = 4x^4
\quad\Longrightarrow\quad
8x^4 - x = 0
\quad\Longrightarrow\quad
x(8x^3-1)=0.
\]
Como $x\geq0$, se obtiene
\[
x= 0 \quad x = \frac{1}{2}.
\]

Sea ahora la recta $y=ax$. Se exige que el área superior $A_1$ sea el doble del área inferior $A_2$:
\[
A_1 = 2A_2.
\]

Dividimos las áreas en función de $x$ según la posición relativa de las curvas:

\[
A_1 = \int_{0}^{b} \left(ax - 2x^2\right)\,dx
\;+\;
\int_{b}^{1/2} \left(\sqrt{\frac{x}{2}} - 2x^2\right)\,dx,
\]

\[
A_2 = \int_{0}^{b} \left(\sqrt{\frac{x}{2}} - ax\right)\,dx.
\]

Calculamos:

\[
A_1
=
\left[
\frac{2x\sqrt{x}}{3\sqrt{2}} - \frac{2x^3}{3}
\right]_{b}^{1/2}
+
\left[
\frac{a x^2}{2} - \frac{2x^3}{3}
\right]_{0}^{b},
\]

\[
A_2
=
\left[
\frac{2x\sqrt{x}}{3\sqrt{2}} - \frac{a x^2}{2}
\right]_{0}^{b}.
\]

Imponiendo $A_1 = 2A_2$ se obtiene la ecuación
\[
\frac{1}{12}
 - \frac{2b\sqrt{b}}{3\sqrt{2}}
 + \frac{ab^2}{2}
=
\frac{4b\sqrt{b}}{3\sqrt{2}}
 - ab^2.
\]

De aquí resulta
\[
\frac{3ab^2}{2}
=
\frac{2b\sqrt{b}}{\sqrt{2}}
 - \frac{1}{12}.
\quad\Longrightarrow\quad a = \frac{4\sqrt{b}}{3b\sqrt{2}}
 - \frac{1}{18b^2}.
\]

Aplicamos que $y(b) = f(b)$:
\[
\frac{4\sqrt{b}}{3\sqrt{2}}
 - \frac{1}{18b} = \sqrt{\frac{b}{2}}.
\qquad\Longrightarrow\quad b = \sqrt[3]{\frac{1}{18}}
\]

Finalmente:
\[
a = \frac{4\sqrt{b}}{3b\sqrt{2}}
 - \frac{1}{18b^2}
\approx 1.1447.
\]

\[
\boxed{a \approx 1.1447}
\]

\section*{Solución Ejercicio 3}
Consideramos la diferencia
\[
A=\int_1^4\left(\frac{1}{x}-\frac{3x}{3x^2+6x+10}\right)\,dx.
\]

Sea
\[
I=\int\frac{3x}{3x^2+6x+10}\,dx.
\]

Escribimos
\[
3x = \frac{1}{2}(6x+6) - \frac{1}{2}\cdot 6.
\]

Así,
\[
I=\frac12\int\frac{6x+6}{3x^2+6x+10}\,dx - \frac12\int\frac{6}{3x^2+6x+10}\,dx.
\]

Observamos que
\[
3x^2+6x+10 = 3(x+1)^2+7.
\]

Luego,
\[
I=\frac12\ln|3x^2+6x+10|
-3\int \frac{1}{7+3(x+1)^2}\,dx.
\]

\[
I=
\frac12\ln|3x^2+6x+10|
-\frac{3\sqrt{7}}{\sqrt{3}\cdot 7}\arctan\!\left(\frac{\sqrt{3}(x+1)}{\sqrt{7}}\right).
\]


\[
A=\left[\ln|x|
-\frac12\ln|3x^2+6x+10|
+\frac{3\sqrt{7}}{\sqrt{3}\cdot 7}
\arctan\!\left(\frac{\sqrt{3}(x+1)}{\sqrt{7}}\right)
\right]_{1}^{4}.
\]

Finalmente: 
\[
A=
\ln 4 - \frac12\ln 82
+\frac{3\sqrt{7}}{\sqrt{3}\cdot 7}\arctan\!\left(\frac{5\sqrt{3}}{\sqrt{7}}\right)
+\ln 1 - \frac12\ln 19
-\frac{3\sqrt{7}}{\sqrt{3}\cdot 7}\arctan\!\left(\frac{\sqrt{3}\cdot 2}{\sqrt{7}}\right),
\]

\[
\boxed{A\approx 0.888}
\]

\section*{Solución Ejercicio 4}
Consideramos
\[
I=\int \frac{x\cos(x)-\sen(x)}{2x^2+\sen^2(x)}\,dx.
\]

Dividimos numerador y denominador entre \(2x^2\):
\[
I=\int \frac{\dfrac{x\cos x}{2x^2}-\dfrac{\sen x}{2x^2}}
              {1+\dfrac{\sen^2 x}{2x^2}}\,dx
   = \int \frac{\dfrac{\cos x}{2x}-\dfrac{\sen x}{2x^2}}
              {1+\left(\dfrac{\sen x}{\sqrt{2}\,x}\right)^2}\,dx.
\]

Observamos que
\[
\frac{d}{dx}\left(\frac{\sen x}{\sqrt{2}\,x}\right)
= \frac{x\cos x-\sen x}{\sqrt{2}\,x^2}
= \sqrt{2}\left(\frac{\cos x}{2x}-\frac{\sen x}{2x^2}\right).
\]

Por tanto,
\[
\frac{\cos x}{2x}-\frac{\sen x}{2x^2}
= \frac{1}{\sqrt{2}}\,
\frac{d}{dx}\left(\frac{\sen x}{\sqrt{2}\,x}\right),
\]

\[
I=\frac{1}{\sqrt{2}}\int 
   \frac{\left(\dfrac{\sen x}{\sqrt{2}\,x}\right)'}
        {1+\left(\dfrac{\sen x}{\sqrt{2}\,x}\right)^2}\,dx,
\]
que es directamente una arctangente:
\[
I=\frac{1}{\sqrt{2}}\arctan\!\left(\frac{\sen x}{\sqrt{2}\,x}\right)+C.
\]

\[
\boxed{\displaystyle
\int \frac{x\cos(x)-\sen(x)}{2x^2+\sen^2(x)}\,dx
= \frac{1}{\sqrt{2}}\arctan\!\left(\frac{\sen x}{\sqrt{2}\,x}\right)+C.}
\]

\section*{Solución Ejercicio 5}
El área pedida viene dada por
\[
A=\int_{-1}^{1} x^{3}\arctan(x)\,dx.
\]

Para ello calculamos primero
\[
I=\int x^{3}\arctan(x)\,dx.
\]

Integramos por partes, tomando
\[
u=\arctan(x),\qquad dv=x^{3}\,dx,
\]
\[
du=\frac{1}{1+x^{2}}\,dx,\qquad v=\frac{x^{4}}{4}.
\]

Entonces
\[
I=\frac{x^{4}}{4}\arctan(x)-\frac{1}{4}\int\frac{x^{4}}{1+x^{2}}\,dx.
\]

\[
\frac{x^{4}}{1+x^{2}} = x^{2}-1+\frac{1}{1+x^{2}},
\]

\[
I=\frac{x^{4}}{4}\arctan(x)-\frac{1}{4}\int\left(x^{2}-1+\frac{1}{1+x^{2}}\right)\!dx.
\]

Así obtenemos
\[
I=\frac{x^{4}}{4}\arctan(x)
-\frac{1}{4}\left(\frac{x^{3}}{3}-x+\arctan(x)\right)+C,
\]

\[
I=\frac{x^{4}}{4}\arctan(x)-\frac{x^{3}}{12}+\frac{x}{4}-\frac{1}{4}\arctan(x)+C.
\]

Luego el área es
\[
A=\left[\frac{x^{4}}{4}\arctan(x)-\frac{x^{3}}{12}+\frac{x}{4}
-\frac{1}{4}\arctan(x)\right]_{-1}^{1}.
\]

\[
\boxed{A=\dfrac{1}{3}}
\]

\section*{Solución Ejercicio 6}
Aplicamos la formula del volumen de revolución:
\[
V=\pi\int_{0}^{2}\frac{1}{(x-3)^2(x+2)}\,dx.
\]

Descomponemos en fracciones simples:
\[
\frac{1}{(x-3)^2(x+2)}
=\frac{A}{x-3}+\frac{B}{(x-3)^2}+\frac{C}{x+2}.
\]

Denominador común:

\[
\frac{A(x-3)(x+2)+B(x+2)+C(x-3)^2}{(x-3)^2(x+2)}
\]

\[
A(x-3)(x+2)+B(x+2)+C(x-3)^2=1.
\]

\[
A = -\frac{1}{25} \quad B=\frac{1}{5} \quad C=\frac{1}{25}.
\]

Por tanto:
\[
\frac{1}{(x-3)^2(x+2)}
= -\frac{1}{25}\frac{1}{x-3}
+ \frac{1}{5}\frac{1}{(x-3)^2}
+ \frac{1}{25}\frac{1}{x+2}.
\]

Integramos:
\[
\int -\frac{1}{25}\frac{dx}{x-3}
= -\frac{1}{25}\ln|x-3|,
\]

\[
\int \frac{1}{5}\frac{dx}{(x-3)^2}
= -\frac{1}{5(x-3)},
\]

\[
\int \frac{1}{25}\frac{dx}{x+2}
= \frac{1}{25}\ln|x+2|.
\]

Por lo que el volumen es: 
\[
V = \pi\left[
\frac{1}{25}\ln|x+2| - \frac{1}{5(x-3)}
- \frac{1}{25}\ln|x-3|
\right]_{0}^{2}.
\]


\[
V=\pi\left(
\frac{1}{25}\ln 6 + \frac{2}{15}
\right).
\]

\[
\boxed{V\approx 0.644}
\]
\section*{Solución Ejercicio 7}
Sea
\[
I=\int_{e}^{e^{2}} \frac{\ln(\ln(x))}{x}\,dx
= \int_{e}^{e^{2}} \ln(\ln(x))\;\frac{1}{x}\,dx.
\]

Integramos por partes:
\[
u=\ln(\ln(x)), \qquad dv=\frac{1}{x}\,dx,
\]
\[
du=\frac{1}{x\ln(x)}\,dx, \qquad v=\ln(x).
\]

Así,
\[
I=\left.\ln(\ln(x))\,\ln(x)\right|_{e}^{e^{2}}
-\int_{e}^{e^{2}} \ln(x)\,\frac{1}{x\ln(x)}\,dx
= \left.\ln(\ln(x))\,\ln(x)\right|_{e}^{e^{2}}
-\int_{e}^{e^{2}} \frac{1}{x}\,dx.
\]

\[
I=\left.\ln(x)\bigl(\ln(\ln(x))-1\bigr)\right|_{e}^{e^{2}}.
\]

Evaluamos en los extremos:
\[
I=\ln(e^{2})\bigl(\ln(\ln(e^{2}))-1\bigr)
-\ln(e)\bigl(\ln(\ln(e))-1\bigr)
=2(\ln 2-1)-1.
\]

Por tanto,
\[
\boxed{
\int_{e}^{e^{2}} \frac{\ln(\ln(x))}{x}\,dx
=2\ln(2)-1
}.
\]

\section*{Solución Ejercicio 8}
Derivamos las funciones paramétricas:
\[
x'(t)=t^2,\qquad y'(t)=t.
\]

Por tanto, la longitud buscada es
\[
L=\int_0^1 \sqrt{(x'(t))^2+(y'(t))^2}\,dt
  =\int_0^1 \sqrt{t^4+t^2}\,dt
  =\int_0^1 t\sqrt{t^2+1}\,dt.
\]

\[
L=\frac{1}{2}\int_{0}^{1} 2t\sqrt{t^{2}+1}\,dt
=\frac{1}{2}\left[\frac{2}{3}(t^{2}+1)^{3/2}\right]_{0}^{1}
  =\frac{1}{3}\left((1^{2}+1)^{3/2}-(0^{2}+1)^{3/2}\right)
\]

\[
\boxed{L=\dfrac{\sqrt{8}-1}{3}\approx 0{,}609}
\]

\section*{Solución Ejercicio 9}
Consideramos
\[
I=\int_{\pi/4}^{\pi/2}\frac{2\sen(x)}{\cos^{2}(x)-3\cos(x)+2}\,dx.
\]

Hacemos el cambio
\[
t=\cos(x)\quad\Rightarrow\quad dt=-\sen(x)\,dx,
\]

Además, cuando \(x=\pi/4\), \(t=\cos(\pi/4)=\dfrac{\sqrt{2}}{2}\) y cuando \(x=\pi/2\),
\(t=\cos(\pi/2)=0\). Entonces
\[
I=\int_{\pi/4}^{\pi/2}\frac{2\sen(x)}{\cos^{2}(x)-3\cos(x)+2}\,dx
   =\int_{\sqrt{2}/2}^{0}\frac{-2}{t^{2}-3t+2}\,dt
   =\int_{0}^{\sqrt{2}/2}\frac{2}{t^{2}-3t+2}\,dt.
\]

Factorizamos el denominador:
\[
t^{2}-3t+2=(t-1)(t-2),
\]
y descomponemos en fracciones simples:
\[
\frac{2}{(t-1)(t-2)}=\frac{A}{t-1}+\frac{B}{t-2}.
\]

De
\[
2=A(t-2)+B(t-1)
\]
tomando \(t=1\) se obtiene \(-B=2\Rightarrow B=-2\), y tomando \(t=2\) se obtiene
\(A=2\). Así,
\[
\frac{2}{(t-1)(t-2)}=\frac{2}{t-1}-\frac{2}{t-2},
\]
Por tanto:
\[
I=\int_{0}^{\sqrt{2}/2}\left(\frac{2}{t-1}-\frac{2}{t-2}\right)dt
  =\left[2\ln|t-1|-2\ln|t-2|\right]_{0}^{\sqrt{2}/2}.
\]

\[
I=
2\ln\left(\frac{1}{2}\right) - 2\ln\left|\frac{\frac{\sqrt{2}}{2}-1}{\frac{\sqrt{2}}{2}-2}\right|.
\]

Una forma equivalente y algo más simple es
\[
\boxed{\displaystyle
I
=2\ln\left(\frac12\right)
-2\ln\left|\frac{\sqrt{2}-2}{\sqrt{2}-4}\right|
}
\]

\[
\boxed{I \approx 1.58}
\]

\section*{Solución Ejercicio 10}
Aplicamos la fórmula del volumen de revolución:
\[
V=\pi\int_{0}^{2\pi} \bigl(f(x)\bigr)^2\,dx
  =\pi\int_{0}^{2\pi} \sen^2(x)\,dx.
\]
Sabemos que
\[
\int_{0}^{2\pi} \sen^2(x)\,dx = \pi,
\]
pues el valor medio de $\sen^2(x)$ en un período es $\tfrac12$.
Por tanto:
\[
\boxed{V=\pi^2}.
\]

\section*{Solución Ejercicio 11}
La masa total es
\[
M = \int_{0}^{4} \lambda(x)\,dx
   = \int_{0}^{4} \left(2+\frac{x}{2}+\frac{1}{\sqrt{x+1}}\right)dx.
\]

Calculamos:
\[
\int 2\,dx = 2x,\qquad
\int \frac{x}{2}\,dx = \frac{x^{2}}{4},\qquad
\int \frac{1}{\sqrt{x+1}}\,dx = 2\sqrt{x+1}.
\]

Por tanto:
\[
M=\left[2x+\frac{x^{2}}{4}+2\sqrt{x+1}\right]_{0}^{4}
  =\bigl(8+4+2\sqrt{5}\bigr)-\bigl(0+0+2\bigr)
  =10+2\sqrt{5}.
\]

\[
\boxed{M=10+2\sqrt{5}\ \text{kg}}.
\]

\section*{Solución Ejercicio 12}
La función $F$ es derivable y
\[
F'(x)=e^{x^{2}}>0\quad\forall x\in\mathbb{R}.
\]
Luego $F$ es estrictamente creciente en todo $\mathbb{R}$, por lo que
\emph{a lo sumo} puede tener una única raíz.

Además,
\[
F(0)=-1+\int_{0}^{0} e^{t^{2}}\,dt=-1<0.
\]

Para $x=1$:
\[
\int_{0}^{1} e^{t^{2}}\,dt > \int_{0}^{1} 1\,dt = 1
\quad\Longrightarrow\quad
F(1)=-1+\int_{0}^{1}e^{t^{2}}\,dt>0.
\]

Como $F$ es continua, por el Teorema del Valor Intermedio existe
$c\in(0,1)$ tal que $F(c)=0$.

Al ser $F$ estrictamente creciente, esa raíz es única.

\[
\boxed{\text{La ecuación }F(x)=0\text{ tiene una única solución }c\in(0,1).}
\]
\end{document}