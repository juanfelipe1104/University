\documentclass[12pt]{article}
\usepackage[utf8]{inputenc}
\usepackage{amsmath, amssymb}
\usepackage{geometry}
\geometry{margin=2.5cm}

\title{Álgebra Lineal \\ Examen Final 2025 \\ Resolución}
\author{}
\date{}

\begin{document}

\maketitle

\section*{Problema 1}

Consideramos el endomorfismo \( f: P_2(x) \to P_2(x) \), y usamos el isomorfismo:
\[
P_2(x) \cong \mathbb{R}^3, \quad a + bx + cx^2 \mapsto (a, b, c)
\]

\subsection*{Datos del problema}

\begin{itemize}
    \item
    \[
    \ker f = \{ (a, b, c) \in \mathbb{R}^3 \mid a = b = c \} = L\langle (1,1,1) \rangle
    \Rightarrow \dim(\ker f) = 1
    \]
    
    \item \(\forall (0,b,c) \in \mathbb{R}^3 \), se tiene:
    \[
    f(0,b,c) = (0,b,c)
    \]
    Es decir:
    \[
    W = \{ (0,b,c) \} = L\langle (0,1,0), (0,0,1) \rangle
    \Rightarrow W \subseteq \operatorname{Im}(f)
    \]
\end{itemize}

\subsection*{a) Matriz del endomorfismo en la base canónica}

Sea la base canónica \( \mathcal{B}_C = \{ e_1 = (1,0,0),\ e_2 = (0,1,0),\ e_3 = (0,0,1) \} \).

Ya sabemos que:

\[
f(e_2) = (0,1,0), \quad f(e_3) = (0,0,1)
\]

Para \( f(e_1) \), sabemos que:

\[
f(e_1) \in \operatorname{Im}(f) = L\langle (0,1,0), (0,0,1) \rangle
\Rightarrow f(1,0,0) = (0,\alpha,\beta)
\]

Ahora usamos linealidad:

\[
f(1,1,1) = f(1,0,0) + f(0,1,0) + f(0,0,1) = (0,\alpha,\beta) + (0,1,0) + (0,0,1) = (0,\alpha + 1, \beta + 1)
\]

Pero como \( (1,1,1) \in \ker f \), entonces:

\[
f(1,1,1) = (0,0,0) \Rightarrow \alpha + 1 = 0,\ \beta + 1 = 0 \Rightarrow \alpha = -1,\ \beta = -1
\]

Por lo tanto:

\[
f(1,0,0) = (0,-1,-1)
\]

Finalmente, la matriz del endomorfismo en la base canónica es:

\[
M_{\mathcal{B}_C}(f) =
\begin{bmatrix}
0 & 0 & 0 \\
-1 & 1 & 0 \\
-1 & 0 & 1 \\
\end{bmatrix}
\]

\subsection*{b) Imagen de \( f \) y su dimensión}

Sabemos que:

\[
\ker f = L\langle (1,1,1) \rangle \Rightarrow \dim(\ker f) = 1
\]

Por el teorema del rango:

\[
\dim(\operatorname{Im} f) = 3 - 1 = 2
\]

Además, como \( f(0,1,0) = (0,1,0) \), \( f(0,0,1) = (0,0,1) \), entonces:

\[
\operatorname{Im} f = L\langle (0,1,0),\ (0,0,1) \rangle
\]

\subsection*{c) Determinar \( f(S) \)}

Sea el conjunto:

\[
S = \left\{ \left( \lambda - 2\mu + \delta,\ \lambda - 2\mu - \delta,\ \lambda - 2\mu \right) \;\middle|\; \lambda, \mu, \delta \in \mathbb{R} \right\}
\]

Aplicamos \( f(a,b,c) = (0,\ -a + b,\ -a + c) \), donde:

\[
a = \lambda - 2\mu + \delta,\quad
b = \lambda - 2\mu - \delta,\quad
c = \lambda - 2\mu
\]

Calculamos los componentes:

\[
-a + b = -(\lambda - 2\mu + \delta) + (\lambda - 2\mu - \delta) = -2\delta
\]
\[
-a + c = -(\lambda - 2\mu + \delta) + (\lambda - 2\mu) = -\delta
\]

Entonces:

\[
f(a,b,c) = (0, -2\delta, -\delta) = -\delta \cdot (0, 2, 1)
\]

Por lo tanto:

\[
f(S) = L\langle (0,2,1) \rangle, \qquad \dim(f(S)) = 1
\]

\subsection*{d) Matriz en la base \( \mathcal{B}^* = \{ (1,1,1),\ (1,1,0),\ (1,0,0) \} \)}

Denotamos:

\[
v_1 = (1,1,1),\quad v_2 = (1,1,0),\quad v_3 = (1,0,0)
\]

Aplicamos la transformación a cada vector:

\[
f(v_1) = f(1,1,1) = (0,0,0) \Rightarrow [f(v_1)]_{\mathcal{B}^*} = \begin{bmatrix} 0 \\ 0 \\ 0 \end{bmatrix}
\]

\[
f(v_2) = f(1,1,0) = f(1,0,0) + f(0,1,0) = (0,-1,-1) + (0,1,0) = (0,0,-1)
\]

Buscamos \( \alpha, \beta, \gamma \) tales que:

\[
\alpha(1,1,1) + \beta(1,1,0) + \gamma(1,0,0) = (0,0,-1)
\]

Resolviendo el sistema:

\[
\begin{cases}
\alpha + \beta + \gamma = 0 \\
\alpha + \beta = 0 \\
\alpha = -1
\end{cases}
\Rightarrow [f(v_2)]_{\mathcal{B}^*} = \begin{bmatrix} -1 \\ 1 \\ 0 \end{bmatrix}
\]

\[
f(v_3) = f(1,0,0) = (0,-1,-1)
\]

Buscamos \( \alpha, \beta, \gamma \) tales que:

\[
\alpha(1,1,1) + \beta(1,1,0) + \gamma(1,0,0) = (0,-1,-1)
\]

Resolviendo:

\[
\begin{cases}
\alpha + \beta + \gamma = 0 \\
\alpha + \beta = -1 \\
\alpha = -1
\end{cases}
\Rightarrow [f(v_3)]_{\mathcal{B}^*} = \begin{bmatrix} -1 \\ 0 \\ 1 \end{bmatrix}
\]

\noindent Finalmente, la matriz del endomorfismo en la base \( \mathcal{B}^* \) es:

\[
M_{\mathcal{B}^*}(f) =
\begin{bmatrix}
0 & -1 & -1 \\
0 & 1 & 0 \\
0 & 0 & 1
\end{bmatrix}
\]


\subsubsection*{Método matricial}

Usamos la fórmula de cambio de base:

\[
M_{\mathcal{B}^*}(f) = 
M_{\mathcal{B}_C \to \mathcal{B}^*} 
\cdot M_{\mathcal{B}_C}(f) 
\cdot M_{\mathcal{B}^* \to \mathcal{B}_C}
\]

Las matrices de cambio de base son:

\[
M_{\mathcal{B}^* \to \mathcal{B}_C} =
\begin{bmatrix}
1 & 1 & 1 \\
1 & 1 & 0 \\
1 & 0 & 0 \\
\end{bmatrix}, \qquad
M_{\mathcal{B}_C \to \mathcal{B}^*} =
\begin{bmatrix}
0 & 0 & 1 \\
0 & 1 & -1 \\
1 & -1 & 0 \\
\end{bmatrix}
\]

La matriz del endomorfismo \( f \) en la base canónica es:

\[
M_{\mathcal{B}_C}(f) =
\begin{bmatrix}
0 & 0 & 0 \\
-1 & 1 & 0 \\
-1 & 0 & 1
\end{bmatrix}
\]

Multiplicando:

\[
M_{\mathcal{B}^*}(f) =
\begin{bmatrix}
0 & 0 & 1 \\
0 & 1 & -1 \\
1 & -1 & 0 \\
\end{bmatrix}
\cdot
\begin{bmatrix}
0 & 0 & 0 \\
-1 & 1 & 0 \\
-1 & 0 & 1
\end{bmatrix}
\cdot
\begin{bmatrix}
1 & 1 & 1 \\
1 & 1 & 0 \\
1 & 0 & 0
\end{bmatrix}
=
\begin{bmatrix}
0 & -1 & -1 \\
0 & 1 & 0 \\
0 & 0 & 1
\end{bmatrix}
\]

Este resultado coincide con el obtenido por coordenadas, como debe ocurrir.

\section*{Problema 2}

Dado un endomorfismo \( f: \mathbb{R}^3 \to \mathbb{R}^3 \) cuya matriz asociada en la base canónica es:

\[
A =
\begin{pmatrix}
4 & -1 & 6 \\
2 & 1 & 6 \\
2 & -1 & 8
\end{pmatrix}
\]

\section*{Diagonalización}

El proceso de diagonalización consiste en encontrar una base B de autovectores tal que la matriz asociada al endomorfismo en dicha base sea diagonal
\[
P^{-1} A P = D
\]

\subsubsection*{Paso 1: Cálculo del polinomio característico}

Calculamos:

\[
\chi_A(\lambda) = \det(A - \lambda I) =
\left|
\begin{array}{ccc}
4 - \lambda & -1 & 6 \\
2 & 1 - \lambda & 6 \\
2 & -1 & 8 - \lambda
\end{array}
\right|
\]

El desarrollo del determinante da:

\[
\chi_A(\lambda) = (2 - \lambda)^2 (9 - \lambda)
\]

Por lo tanto, los valores propios son:

\[
\lambda_1 = 2 \quad \text{(multiplicidad algebraica 2)}, \quad
\lambda_2 = 9 \quad \text{(multiplicidad algebraica 1)}
\]

\subsubsection*{Paso 2: Cálculo de los subespacios propios}

\textbf{Para \( \lambda = 2 \)}:

\[
A - 2I =
\begin{pmatrix}
2 & -1 & 6 \\
2 & -1 & 6 \\
2 & -1 & 6
\end{pmatrix}
\Rightarrow \text{Rango 1}
\Rightarrow \dim(\ker(A - 2I)) = 2
\]

\textbf{Base del subespacio propio \( E_2 \):}

Al resolver \( (A - 2I)\vec{v} = 0 \), obtenemos una base:

\[
E_2 = L\langle (1, 2, 0),\ (0, 6, 1) \rangle
\]

\textbf{Para \( \lambda = 9 \)}:

\[
A - 9I =
\begin{pmatrix}
-5 & -1 & 6 \\
2 & -8 & 6 \\
2 & -1 & -1
\end{pmatrix}
\Rightarrow \text{Rango 2}
\Rightarrow \dim(\ker(A - 9I)) = 1
\]

\textbf{Base de \( E_9 \):}

Resolviendo \( (A - 9I)\vec{v} = 0 \) se obtiene:

\[
E_9 = L\langle (1, 1, 1) \rangle
\]

\subsubsection*{Paso 3: Matrices \( P \) y \( D \)}

Tomamos como base los vectores propios:

\[
v_1 = (1, 2, 0), \quad
v_2 = (0, 6, 1), \quad
v_3 = (1, 1, 1)
\]

Entonces:

\[
P = \begin{pmatrix}
1 & 0 & 1 \\
2 & 6 & 1 \\
0 & 1 & 1
\end{pmatrix}, \quad
D = \begin{pmatrix}
2 & 0 & 0 \\
0 & 2 & 0 \\
0 & 0 & 9
\end{pmatrix}
\]

Y se cumple:

\[
P^{-1} A P = D
\]

\section*{Cuestión 1}

Calcular la dimensión y una base del subespacio:

\[
U = \left\{ X \in M_{2 \times 2}(\mathbb{R}) \,\middle|\, \operatorname{tr}(MX) = 0 \right\}, \quad \text{con } M =
\begin{pmatrix}
1 & -1 \\
-1 & 1
\end{pmatrix}
\]

\subsection*{Resolución}

Sea
\[
X =
\begin{pmatrix}
a & b \\
c & d
\end{pmatrix}
\in M_{2 \times 2}(\mathbb{R})
\]

Entonces:
\[
MX =
\begin{pmatrix}
1 & -1 \\
-1 & 1
\end{pmatrix}
\begin{pmatrix}
a & b \\
c & d
\end{pmatrix}
=
\begin{pmatrix}
a - c & b - d \\
- a + c & -b + d
\end{pmatrix}
\]

\[
\operatorname{tr}(MX) = (a - c) + (-b + d) = a - b - c + d
\]

\[
\operatorname{tr}(MX) = 0 \rightarrow a - b - c + d = 0
\]

La dimensión del subespacio \( U \) es:

\[
\dim(U) = 3
\]

y una base está dada por las matrices:

\[
\left\{
\begin{pmatrix}
1 & 1 \\
0 & 0
\end{pmatrix},
\begin{pmatrix}
0 & -1 \\
1 & 0
\end{pmatrix},
\begin{pmatrix}
0 & 1 \\
0 & 1
\end{pmatrix}
\right\}
\]

\section*{Cuestión 2}

Dadas las aplicaciones lineales:

\begin{itemize}
    \item \( f: \mathbb{R}^2 \to \mathbb{R}^3 \), tal que:
    \[
    f(1,0) = (2,1,2), \quad f(0,1) = (3,2,1)
    \]
    \item \( g: \mathbb{R}^3 \to \mathbb{R}^2 \), tal que:
    \[
    g(1,0,0) = (3,-1), \quad g(0,1,0) = (1,0), \quad g(0,0,1) = (1,-2)
    \]
\end{itemize}

Definimos la aplicación \( h = g \circ f: \mathbb{R}^2 \to \mathbb{R}^2 \) y queremos calcular \( h(4,-2) \) por dos métodos: por coordenadas y matricialmente.

\subsection*{Por coordenadas}

Primero calculamos \( f(4,-2) \) como combinación lineal:

\[
f(4,-2) = 4 \cdot f(1,0) + (-2) \cdot f(0,1)
= 4(2,1,2) - 2(3,2,1) = (8,4,8) - (6,4,2) = (2,0,6)
\]

Luego aplicamos \( g \):

\[
g(2,0,6) = 2 \cdot g(1,0,0) + 6 \cdot g(0,0,1)
= 2(3,-1) + 6(1,-2) = (6,-2) + (6,-12) = (12, -14)
\]

\[
\boxed{h(4,-2) = (12, -14)}
\]

\subsection*{Matricialmente}

La matriz de \( f \) (por columnas \( f(1,0), f(0,1) \)) es:

\[
M_f =
\begin{pmatrix}
2 & 3 \\
1 & 2 \\
2 & 1
\end{pmatrix}
\]

La matriz de \( g \) (por filas \( g \) aplicado a las canónicas de \( \mathbb{R}^3 \)) es:

\[
M_g =
\begin{pmatrix}
3 & 1 & 1 \\
-1 & 0 & -2
\end{pmatrix}
\]

La matriz de la composición es:

\[
M_h = M_g M_f =
\begin{pmatrix}
3 & 1 & 1 \\
-1 & 0 & -2
\end{pmatrix}
\begin{pmatrix}
2 & 3 \\
1 & 2 \\
2 & 1
\end{pmatrix}
=
\begin{pmatrix}
9 & 12 \\
-6 & -5
\end{pmatrix}
\]

Finalmente:

\[
h(4,-2) = M_h \cdot 
\begin{pmatrix}
4 \\ -2
\end{pmatrix}
=
\begin{pmatrix}
9 & 12 \\
-6 & -5
\end{pmatrix}
\begin{pmatrix}
4 \\ -2
\end{pmatrix}
=
\begin{pmatrix}
12 \\
-14
\end{pmatrix}
\]

\[
\boxed{h(4,-2) = (12, -14)}
\]

\section*{Cuestión 3}

Dado:

\[
U = L\langle (1,0,0,1),\ (1,1,1,1),\ (0,2,2,0) \rangle,\quad
W = \left\{ (x,y,z,t) \in \mathbb{R}^4 \;\middle|\; 
\begin{array}{l}
3x + y - z - 3t = 0 \\
y - t = 0
\end{array}
\right\}
\]

\subsection*{Paso 1: Dimensiones}

Notamos que:

\[
(1,1,1,1) = (1,0,0,1) + \frac{1}{2}(0,2,2,0)
\Rightarrow \text{Los vectores de } U \text{ son linealmente dependientes}
\]

Una base de \( U \) es:

\[
\{ (1,0,0,1),\ (0,2,2,0) \} \Rightarrow \dim(U) = 2
\]

Resolvemos el sistema de \( W \):

\[
y = t,\quad 3x + y - z - 3t = 0 \Rightarrow z = 3x - 2t
\]

\[
\Rightarrow W = \left\{ x(1,0,3,0) + t(0,1,-2,1) \right\}
\Rightarrow \dim(W) = 2
\]

\subsection*{Paso 2: Verificación de suma directa}

Juntamos las bases:

\[
\left\{
(1,0,0,1),\ (0,2,2,0),\ (1,0,3,0),\ (0,1,-2,1)
\right\}
\]

Formamos la matriz con estos vectores como filas y verificamos que el rango es 4. Por tanto:

\[
\dim(U + W) = 4 = \dim(U) + \dim(W) \Rightarrow U \cap W = \{0\}
\]

\subsection*{Conclusión}

\[
\boxed{U + W \text{ es una suma directa}}
\]

\section*{Cuestión 4}

\textbf{1.} Una matriz \( A \) tiene autovalor 0 si y solo si \( \det(A) = 0 \)

\textbf{Razonamiento:}

Si \( A \) tiene autovalor 0, entonces existe un vector no nulo \( v \) tal que \( Av = 0 \), lo cual implica que el sistema homogéneo tiene solución no trivial y por tanto \( \det(A) = 0 \).

Si \( \det(A) = 0 \), entonces el sistema \( Av = 0 \) tiene infinitas soluciones  \( \rightarrow 0 \) es autovalor.

\[
\boxed{\text{Verdadero}}
\]

\textbf{2.} Las matrices

\[
A =
\begin{pmatrix}
1 & 2 & 0 \\
-1 & 3 & 1 \\
0 & 1 & 1
\end{pmatrix}, \quad
B =
\begin{pmatrix}
5 & 0 & -4 \\
0 & 3 & 0 \\
2 & 0 & -1
\end{pmatrix}
\]

son semejantes.

\textbf{Razonamiento:}

Dos matrices son semejantes si;

- Tienen el mismo rango

- Tienen el mismo determinante

- Tienen la misma traza

Calculamos las trazas:

\[
tr(A) = 1 + 3 + 1 = 5 \quad tr(B) = 5 + 3 - 1 = 7 \quad 5 \neq 7
\]

\[
\boxed{\text{Falso}}
\]

\end{document}